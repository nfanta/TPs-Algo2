\documentclass[a4paper,10pt]{article}
\usepackage[paper=a4paper, hmargin=1.5cm, bottom=1.5cm, top=3.5cm]{geometry}
\usepackage[latin1]{inputenc}
\usepackage[T1]{fontenc}
\usepackage[spanish]{babel}
\usepackage{xspace}
\usepackage{xargs}
\usepackage{ifthen}
\usepackage{aed2-tad,aed2-symb,aed2-itef}
\usepackage[]{algorithm2e}

\SetKwInput{KwData}{Complejidad}
\SetKwInput{KwComm}{Comentarios}
\SetKw{ret}{return}
\SetKw{var}{var \normalfont}
\SetAlgorithmName{Algoritmo}{Algoritmo}{\large Lista de algoritmos}
\SetAlCapSkip{1em}

\newcommand{\moduloNombre}[1]{\textbf{#1}}

\let\NombreFuncion=\textsc
\let\TipoVariable=\texttt
\let\ModificadorArgumento=\textbf
\newcommand{\res}{$res$\xspace}
\newcommand{\tab}{\hspace*{7mm}}

\newcommandx{\TipoFuncion}[3]{%
  \NombreFuncion{#1}(#2) \ifx#3\empty\else $\to$ \res\,: \TipoVariable{#3}\fi%
}
\newcommand{\In}[2]{\ModificadorArgumento{in} \ensuremath{#1}\,: \TipoVariable{#2}\xspace}
\newcommand{\Out}[2]{\ModificadorArgumento{out} \ensuremath{#1}\,: \TipoVariable{#2}\xspace}
\newcommand{\Inout}[2]{\ModificadorArgumento{in/out} \ensuremath{#1}\,: \TipoVariable{#2}\xspace}
\newcommand{\Aplicar}[2]{\NombreFuncion{#1}(#2)}

\newlength{\IntFuncionLengthA}
\newlength{\IntFuncionLengthB}
\newlength{\IntFuncionLengthC}
%InterfazFuncion(nombre, argumentos, valor retorno, precondicion, postcondicion, complejidad, descripcion, aliasing)
\newcommandx{\InterfazFuncion}[9][4=true,6,7,8,9]{%
  \hangindent=\parindent
  \TipoFuncion{#1}{#2}{#3}\\%
  \textbf{Pre} $\equiv$ \{#4\}\\%
  \textbf{Post} $\equiv$ \{#5\}%
  \ifx#6\empty\else\\\textbf{Complejidad:} #6\fi%
  \ifx#7\empty\else\\\textbf{Descripci�n:} #7\fi%
  \ifx#8\empty\else\\\textbf{Aliasing:} #8\fi%
  \ifx#9\empty\else\\\textbf{Requiere:} #9\fi%
}

\newenvironment{Interfaz}{%
  \parskip=2ex%
  \noindent\textbf{\Large Interfaz}%
  \par%
}{}

\newenvironment{Representacion}{%
  \vspace*{2ex}%
  \noindent\textbf{\Large Representaci�n}%
  \vspace*{2ex}%
}{}

\newenvironment{Algoritmos}{%
  \vspace*{2ex}%
  \noindent\textbf{\Large Algoritmos}%
  \vspace*{2ex}%
}{}


\newcommand{\Titulo}[1]{
  \vspace*{1ex}\par\noindent\textbf{\large #1}\par
}

\newenvironmentx{Estructura}[2][2={estr}]{%
  \par\vspace*{2ex}%
  \TipoVariable{#1} \textbf{se representa con} \TipoVariable{#2}%
  \par\vspace*{1ex}%
}{%
  \par\vspace*{2ex}%
}%

\newboolean{EstructuraHayItems}
\newlength{\lenTupla}
\newenvironmentx{Tupla}[1][1={estr}]{%
    \settowidth{\lenTupla}{\hspace*{3mm}donde \TipoVariable{#1} es \TipoVariable{tupla}$($}%
    \addtolength{\lenTupla}{\parindent}%
    \hspace*{3mm}donde \TipoVariable{#1} es \TipoVariable{tupla}$($%
    \begin{minipage}[t]{\linewidth-\lenTupla}%
    \setboolean{EstructuraHayItems}{false}%
}{%
    $)$%
    \end{minipage}
}

\newcommandx{\tupItem}[3][1={\ }]{%
    %\hspace*{3mm}%
    \ifthenelse{\boolean{EstructuraHayItems}}{%
        ,#1%
    }{}%
    \emph{#2}: \TipoVariable{#3}%
    \setboolean{EstructuraHayItems}{true}%
}

\newcommandx{\RepFc}[3][1={estr},2={e}]{%
  \tadOperacion{Rep}{#1}{bool}{}%
  \tadAxioma{Rep($#2$)}{#3}%
}%

\newcommandx{\Rep}[3][1={estr},2={e}]{%
  \tadOperacion{Rep}{#1}{bool}{}%
  \tadAxioma{Rep($#2$)}{true \ssi #3}%
}%

\newcommandx{\Abs}[5][1={estr},3={e}]{%
  \tadOperacion{Abs}{#1/#3}{#2}{Rep($#3$)}%
  \settominwidth{\hangindent}{Abs($#3$) \igobs #4: #2 $\mid$ }%
  \addtolength{\hangindent}{\parindent}%
  Abs($#3$) \igobs #4: #2 $\mid$ #5%
}%

\newcommandx{\AbsFc}[4][1={estr},3={e}]{%
  \tadOperacion{Abs}{#1/#3}{#2}{Rep($#3$)}%
  \tadAxioma{Abs($#3$)}{#4}%
}%


\newcommand{\DRef}{\ensuremath{\rightarrow}}



%%%%%%%%%%%%%%%%%%%%%%%%%%%%
%% COMIENZO DEL DOCUMENTO %%
%%%%%%%%%%%%%%%%%%%%%%%%%%%%



\begin{document}



%%%%%%%%%%%%%%%%%%%
%% MODULO CAMPUS %%
%%%%%%%%%%%%%%%%%%%



\section{M�dulo DiccClavesRapidas}


\begin{Interfaz}
  
  \textbf{par�metros formales}\hangindent=2\parindent\\
  \parbox{1.7cm}{\textbf{g�nero}} $significado$\\
  \parbox[t]{1.7cm}{\textbf{funci�n}}\parbox[t]{\textwidth-2\parindent-1.7cm}{%
    \InterfazFuncion{Copiar}{\In{a}{$significado$}}{$significado$}
    {$res \igobs a$}
    [$\Theta(copy(a))$]
    [funci�n de copia de $significado$]
  }

  \textbf{se explica con}: \tadNombre{Diccionario(string, significado)}

  \textbf{g�neros}: \TipoVariable{dcr}
  \bigskip
  \newline
  El modulo funciona como un diccionario, pero solo se utiliza con claves del tipo $string$.
  

  \Titulo{Operaciones b�sicas de DiccClavesRapidas}

  \InterfazFuncion{Vacio}{}{dcr}
  [true]
  {$res \igobs$ vacio}
  [$O(1)$]
  [Crea un diccionario vacio.]

  \InterfazFuncion{Definir}{\In{n}{string}, \In{s}{significado}, \Inout{d}{dcr}}{}
  [$d \igobs d_{0}$]
  {$d \igobs$ definir($n, s, d_{0}$)}
  [$O(Longitud(n))$]
  [Define la clave $n$ con significado $s$ en el diccionario $d$.]
  []

  \InterfazFuncion{Def?}{\In{n}{string}, \In{d}{dcr}}{bool}
  [true]
  {$res \igobs$ def?($n, d$)}
  [$O(Longitud(n))$]
  [Devuelve $true$ si el string $n$ esta definido en el diccionario $d$.]
  []

  \InterfazFuncion{Obtener}{\In{s}{string}, \In{d}{dcr}}{significado}
  [def?($n, d$)]
  {$res \igobs$ obtener($p, m$) $\land$ alias($res, obtener(p, n)$)}
  [$O(Longitud(n))$]
  [Retorna el $significado$ de la clave $n$.]
  [res devuelve el significado por referencia.]

  \InterfazFuncion{Borrar}{\In{s}{string}, \Inout{d}{dcr}}{}
  [def?($p, m$) $\land$ $d \igobs d_{0}$]
  {$d \igobs$ borrar($p, d_{0}$)}
  [$O(Longitud(n))$]
  [Elimina la clave $n$ y su $significado$ del diccionario $d$.]
  []

  \InterfazFuncion{Claves}{\In{d}{dcr}}{conj(string)}
  [true]
  {$res \igobs$ claves($d$) $\land$ alias($res \igobs claves(d)$)}
  [$O(1)$]
  [Devuelve el conjunto de claves del diccionario $d$]
  [$res$ no es modificable. Si se definen nuevas claves, $res$ se invalida.]

\end{Interfaz}
\bigskip


\begin{Representacion}
  
  \Titulo{Representaci�n de DiccClavesRapidas}

  \begin{Estructura}{dcr}[estr]
    \begin{Tupla}[estr]
      \tupItem{dicc}{arreglo(puntero(nodo))}
      \tupItem{claves}{conjunto(string)}
    \end{Tupla}

    \begin{Tupla}[nodo]
      \tupItem{id}{char}
      \tupItem{clave}{string}
      \tupItem{definido}{bool}
      \tupItem{dato}{significado}
      \tupItem{sig}{arreglo(puntero(nodo))}%
    \end{Tupla}


  \end{Estructura}
  \bigskip
  \textbf{Invariante de representacion en castellano:}
  \begin{enumerate}
  
  \item Toda clave de $claves$ esta definida en $dicc$, es decir, $dicc$ posee un nodo con la misma clave.
  \item Todo nodo donde su .definido es $true$, su .clave esta contenida en $claves$.


  \end{enumerate}

  \bigskip
  \Rep[estr][e]{
  
  \begin{enumerate}
  \item ($\forall$ s : string) $s$ $\in$ e.claves $\implies$ ( hayCamino($s$, e.dicc) $\yluego$ $\Pi_{3}$(tuplaEnDicc($s$, e.dicc)) ) $\land$
  \item ($\forall$ s : string) (hayCamino($s$, e.dicc) $\yluego$ $\Pi_{3}$(tuplaEnDicc($s$, e.dicc)) $\implies$ $s$ $\in$ e.claves
  \end{enumerate}
  }

  

  \bigskip
  
  \AbsFc[estr]{dicc($string, significado$)}[e]{d : dicc($string, significado$) \\
    ($\forall$ $s$ : string) def?($s$, e.claves) $\igobs$ def?($p, d$) $\land$ \\
    ($\forall$ $s$ : string) def?($s$, e.claves) $\impluego$ $\Pi_{4}$(tuplaEnDicc($s$, e.dicc)) $\igobs$ obtener($s, d$))
  }
  \bigskip

  \textbf{Operaciones auxiliares de TAD:}
  \smallskip

  \tadOperacion{hayCamino}{string \ $s$, arreglo \ $array$}{bool}{tam(array) = 256}
  \tadAxioma{hayCamino($s$, e.dicc)}{
    \IF {Longitud($s$) = 0} 
      THEN {$true$}
      ELSE {
        \IF {definido(ord(prim($s$)), array)} 
          THEN {hayCamino(fin($s$), array[ord(prim($s$))] )}
          ELSE {$false$} 
        FI} 
    FI
  }

  \bigskip
  \tadOperacion{tuplaEnDicc}{string \ $s$, arreglo \ $array$}{tupla($\alpha_{1} ... \alpha_{n}$)}{hayCamino($s, array$) $\land$ tam(array) = 256}
  \tadAxioma{tuplaEnDicc($s$, e.dicc)}{
    \IF {Longitud($s$) = 1} 
      THEN {array[ord(prim($s$))]}
      ELSE {
        tuplaEnDicc(fin($s$), array[ord(prim($s$))]
      }
    FI
  }
  

\end{Representacion}
\newpage
\begin{Algoritmos}

  % \textbf{Algoritmos de Campus}
  \listofalgorithms
  \bigskip
  

  \begin{algorithm}[H]
    \text{$i$Vacio() $\rightarrow$ res: estr}
  
    \Begin{
      res.dicc $\leftarrow$ CrearArreglo(256) \hfill \textbf{//O(1)}\\
      res.claves $\leftarrow$ Vacio() \hfill \textbf{//O(1)}\\
      \ret{res}
    }
    \KwData{$O(1)$}
    
    \caption{Vacio}
  \end{algorithm}
  \rule{17.5cm}{0.4pt}
  \bigskip


  \begin{algorithm}[H]
    \text{$i$Definir(\In{n}{string}, \In{s}{significado}, \Inout{e}{estr})}
  
    \Begin{
      \var{\normalfont t : string; p : puntero(nodo) tuplaVacia : nodo}\\
      t $\leftarrow$ fin(n) \hfill \textbf{//O(1)}\\
      tuplaVacia $\leftarrow$ <$false$, $s$, CrearArreglo(256)> \hfill \textbf{//O(1)}\\

      \If(\hfill \textbf{//O()}){$\neg$Def?(n, e.claves)}
      {

      \If(\hfill \textbf{//O(1)}){$\neg$Definido(e.dicc, Ord(Prim(t)) )}
        {e.dicc[Ord(Prim(t))] $\leftarrow$ tuplaVacia} \hfill \textbf{//O(1)}

      p $\leftarrow$ e.dicc[Ord(Prim(t))] \hfill \textbf{//O(1)}\\

      \While(\hfill \textbf{//O(Longitud(n))}){Longitud(t) > 0}{
        \If(\hfill \textbf{//O(1)}){$\neg$Definido(p.sig, Ord(Prim(t)) )}
          {p.sig[Ord(Prim(t))] $\leftarrow$ tuplaVacia}\hfill \textbf{//O(1)}

        p $\leftarrow$ p[Ord(Prim(t))] \hfill \textbf{//O(1)}\\
        t $\leftarrow$ fin(t) \hfill \textbf{//O(1)}\\
      }

      p.definido $\leftarrow$ $true$ \hfill \textbf{//O(1)}\\
      p.dato $\leftarrow$ $S$ \hfill \textbf{//O(1)}\\
      Agregar(n, e.claves) \hfill \textbf{//O(1)}
      }
    }
    \KwData{O(Longitud(n))}
    \KwComm{$Agregar(n, e.claves)$ es O(1) dado que se sabe por el if que no esta definido.}
  
    \caption{Definir}
  \end{algorithm}
  \rule{17.5cm}{0.4pt}
  \bigskip


  \begin{algorithm}[H]
    \text{$i$Def?(\In{n}{string}, \In{e}{estr}) $\rightarrow$ res: bool}
  
    \Begin{
      res $\leftarrow$ Def?(n, e.claves) \hfill \textbf{//O(1)}
      \ret{res}
    }
    \KwData{O(1)}
  
    \caption{Def?}
  \end{algorithm}
  \rule{17.5cm}{0.4pt}
  \bigskip


  \begin{algorithm}[H]
    \text{$i$Obtener(\In{n}{string}, \In{e}{estr}) $\rightarrow$ res: significado}
  
    \Begin{
      \var{\normalfont t : string; p : puntero(nodo)}\\
      t $\leftarrow$ fin(n) \hfill \textbf{//O(1)}\\
      p $\leftarrow$ e.dicc[prim(n)] \hfill \textbf{//O(1)}\\

      \While(\hfill \textbf{//O(Longitud(n))}){Longitud(t) > 0}{
        p $\leftarrow$ p[Ord(Prim(t))] \hfill \textbf{//O(1)}\\
        t $\leftarrow$ fin(t) \hfill \textbf{//O(1)}\\
      }

      res $\leftarrow$ p.dato
      \ret{res}
    }
    \KwData{O(Longitud(n))}
  
    \caption{Obtener}
  \end{algorithm}
  \rule{17.5cm}{0.4pt}
  \bigskip



  \begin{algorithm}[H]
    \text{$i$Borrar(\In{s}{string}, \Inout{d}{dcr})}
  
    \Begin{
      \var{\normalfont t : string; p : puntero(nodo)}\\
      t $\leftarrow$ fin(n) \hfill \textbf{//O(1)}\\
      p $\leftarrow$ e.dicc[prim(n)] \hfill \textbf{//O(1)}\\

      \While(\hfill \textbf{//O(Longitud(n))}){Longitud(t) > 0}{
        p $\leftarrow$ p[Ord(Prim(t))] \hfill \textbf{//O(1)}\\
        t $\leftarrow$ fin(t) \hfill \textbf{//O(1)}\\
      }

      p.definido $\leftarrow$ $false$
    }
    \KwData{O(Longitud(n))}
  
    \caption{Borrar}
  \end{algorithm}
  \rule{17.5cm}{0.4pt}
  \bigskip



  \begin{algorithm}[H]
    \text{$i$Claves(\In{e}{estr}) $\rightarrow$ res: conj(string)}
  
    \Begin{
      res $\leftarrow$ e.claves
    }
    \KwData{O(1)}
  
    \caption{Claves}
  \end{algorithm}
  \rule{17.5cm}{0.4pt}
  \bigskip


\end{Algoritmos}

\end{document}