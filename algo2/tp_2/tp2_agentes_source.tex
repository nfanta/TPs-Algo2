\documentclass[a4paper,10pt]{article}
\usepackage[paper=a4paper, hmargin=1.5cm, bottom=1.5cm, top=3.5cm]{geometry}
\usepackage[latin1]{inputenc}
\usepackage[T1]{fontenc}
\usepackage[spanish]{babel}
\usepackage{xspace}
\usepackage{xargs}
\usepackage{ifthen}
\usepackage{aed2-tad,aed2-symb,aed2-itef}
\usepackage[]{algorithm2e}

\SetKwInput{KwData}{Complejidad}
\SetKwInput{KwComm}{Comentarios}
\SetKwInput{KwExp}{Explicacion}
\SetAlgorithmName{Algoritmo}{Algoritmo}{Lista de algoritmos}
\SetAlCapSkip{1em}


\newcommand{\moduloNombre}[1]{\textbf{#1}}

\let\NombreFuncion=\textsc
\let\TipoVariable=\texttt
\let\ModificadorArgumento=\textbf
\newcommand{\res}{$res$\xspace}
\newcommand{\tab}{\hspace*{7mm}}

\newcommandx{\TipoFuncion}[3]{%
  \NombreFuncion{#1}(#2) \ifx#3\empty\else $\to$ \res\,: \TipoVariable{#3}\fi%
}
\newcommand{\In}[2]{\ModificadorArgumento{in} \ensuremath{#1}\,: \TipoVariable{#2}\xspace}
\newcommand{\Out}[2]{\ModificadorArgumento{out} \ensuremath{#1}\,: \TipoVariable{#2}\xspace}
\newcommand{\Inout}[2]{\ModificadorArgumento{in/out} \ensuremath{#1}\,: \TipoVariable{#2}\xspace}
\newcommand{\Aplicar}[2]{\NombreFuncion{#1}(#2)}

\newlength{\IntFuncionLengthA}
\newlength{\IntFuncionLengthB}
\newlength{\IntFuncionLengthC}
%InterfazFuncion(nombre, argumentos, valor retorno, precondicion, postcondicion, complejidad, descripcion, aliasing)
\newcommandx{\InterfazFuncion}[9][4=true,6,7,8,9]{%
  \hangindent=\parindent
  \TipoFuncion{#1}{#2}{#3}\\%
  \textbf{Pre} $\equiv$ \{#4\}\\%
  \textbf{Post} $\equiv$ \{#5\}%
  \ifx#6\empty\else\\\textbf{Complejidad:} #6\fi%
  \ifx#7\empty\else\\\textbf{Descripci�n:} #7\fi%
  \ifx#8\empty\else\\\textbf{Aliasing:} #8\fi%
  \ifx#9\empty\else\\\textbf{Requiere:} #9\fi%
}

\newenvironment{Interfaz}{%
  \parskip=2ex%
  \noindent\textbf{\Large Interfaz}%
  \par%
}{}

\newenvironment{Representacion}{%
  \vspace*{2ex}%
  \noindent\textbf{\Large Representaci�n}%
  \vspace*{2ex}%
}{}

\newenvironment{Algoritmos}{%
  \vspace*{2ex}%
  \noindent\textbf{\Large Algoritmos}%
  \vspace*{2ex}%
}{}


\newcommand{\Titulo}[1]{
  \vspace*{1ex}\par\noindent\textbf{\large #1}\par
}

\newenvironmentx{Estructura}[2][2={estr}]{%
  \par\vspace*{2ex}%
  \TipoVariable{#1} \textbf{se representa con} \TipoVariable{#2}%
  \par\vspace*{1ex}%
}{%
  \par\vspace*{2ex}%
}%

\newboolean{EstructuraHayItems}
\newlength{\lenTupla}
\newenvironmentx{Tupla}[1][1={estr}]{%
    \settowidth{\lenTupla}{\hspace*{3mm}donde \TipoVariable{#1} es \TipoVariable{tupla}$($}%
    \addtolength{\lenTupla}{\parindent}%
    \hspace*{3mm}donde \TipoVariable{#1} es \TipoVariable{tupla}$($%
    \begin{minipage}[t]{\linewidth-\lenTupla}%
    \setboolean{EstructuraHayItems}{false}%
}{%
    $)$%
    \end{minipage}
}

\newcommandx{\tupItem}[3][1={\ }]{%
    %\hspace*{3mm}%
    \ifthenelse{\boolean{EstructuraHayItems}}{%
        ,#1%
    }{}%
    \emph{#2}: \TipoVariable{#3}%
    \setboolean{EstructuraHayItems}{true}%
}

\newcommandx{\RepFc}[3][1={estr},2={e}]{%
  \tadOperacion{Rep}{#1}{bool}{}%
  \tadAxioma{Rep($#2$)}{#3}%
}%

\newcommandx{\Rep}[3][1={estr},2={e}]{%
  \tadOperacion{Rep}{#1}{bool}{}%
  \tadAxioma{Rep($#2$)}{true \ssi #3}%
}%

\newcommandx{\Abs}[5][1={estr},3={e}]{%
  \tadOperacion{Abs}{#1/#3}{#2}{Rep($#3$)}%
  \settominwidth{\hangindent}{Abs($#3$) \igobs #4: #2 $\mid$ }%
  \addtolength{\hangindent}{\parindent}%
  Abs($#3$) \igobs #4: #2 $\mid$ #5%
}%

\newcommandx{\AbsFc}[4][1={estr},3={e}]{%
  \tadOperacion{Abs}{#1/#3}{#2}{Rep($#3$)}%
  \tadAxioma{Abs($#3$)}{#4}%
}%


\newcommand{\DRef}{\ensuremath{\rightarrow}}

\begin{document}


\section{Modulo Agentes}


\begin{Interfaz}
  
  \textbf{se explica con}: \tadNombre{Agentes}

  \textbf{generos}: \TipoVariable{agentes}

  \textbf{usa}: \TipoVariable{conj($\alpha$)}, \TipoVariable{diccRapido(clave, significado)}, \TipoVariable{nat}, \TipoVariable{bool}, \TipoVariable{lista}

  \Titulo{Operaciones basicas de agentes}

  \InterfazFuncion{NuevoAgentes}{\In{as}{dicc(placa, posicion)}}{agentes}%
  [$\neg$ $\emptyset$?(claves($as$))]
  {$res \igobs$ nuevoAgentes($as$)}%
  [O(N$a$) //Revisar al hacer algoritmo]
  [Crea un nuevo contenedor de Agentes con los agentes contenidos en $as$. N$a$ es la cantidad de agentes definidos en $as$]

  \InterfazFuncion{Agentes?}{\In{as}{agentes}}{conj(placa)}%
  [true]
  {$res \igobs$ agentes?($as$)}%
  [$\Theta(1)$]
  [Me devuelve un conjunto de todos los agentes definidos en $as$]
  [$res$ es una referencia constante a un conj(placa).]

  \InterfazFuncion{AgregarSancion}{\In{a}{placa}, \Inout{as}{agentes}}{}%
  [estaAgente($a, as$) $\land$ as = $as_{0}$]
  {$as \igobs$ agregarSancion($a, as_{0}$)}%
  [O(N$a$) en peor caso. O(1) si se asegura distribucion uniforme de las placas]
  [Agrega una sancion al agente $a$]

  \InterfazFuncion{CambiarPosicion}{\In{a}{placa}, \In{p}{posicion}, \Inout{as}{agentes}}{}%
  [estaAgente($a, as$) $\land$ as = $as_{0}$]
  {$as \igobs$ cambiarPos($a, p, as_{0}$)}%
  [O(N$a$) en peor caso. O(1) si se asegura distribucion uniforme de las placas]
  [Modifica la posicion del agente $a$, para que sea $p$]

  \InterfazFuncion{AgregarCaptura}{\In{a}{placa}, \Inout{as}{agentes}}{}%
  [estaAgente($a, as$) $\land$ as = $as_{0}$]
  {$as \igobs$ agregarCaptura($a, as_{0}$)}%
  [O(N$a$) en peor caso. O(1) si se asegura distribucion uniforme de las placas]
  [Agrega una captura al agente $a$]

  \InterfazFuncion{PosAgente}{\In{a}{placa}, \In{as}{agentes}}{posicion}%
  [estaAgente($a, as$)]
  {$res \igobs$ posicionAgente($a, as$)}%
  [O(N$a$) en peor caso. O(1) si se asegura distribucion uniforme de las placas]
  [Devuelve la posicion actual del agente $a$]

  \InterfazFuncion{SancionesAgente}{\In{a}{placa}, \In{as}{agentes}}{nat}%
  [estaAgente($a, as$)]
  {$res \igobs$ sancionesAgente($a, as$)}%
  [O(N$a$) en peor caso. O(1) si se asegura distribucion uniforme de las placas]
  [Devuelve las sanciones actuales del agente $a$]

  \InterfazFuncion{CapturasAgente}{\In{a}{placa}, \In{as}{agentes}}{nat}%
  [estaAgente($a, as$)]
  {$res \igobs$ capturasAgente($a, as$)}%
  [O(N$a$) en peor caso. O(1) si se asegura distribucion uniforme de las placas]
  [Devuelve la cantidad de capturas actuales del agente $a$]

  \InterfazFuncion{MasVigilante}{\In{as}{agentes}}{placa}%
  [true]
  {$res \igobs$ masVigilante($as$)}%
  [$\Theta(1)$]
  [Devuelve el agente que mas capturas tiene en $as$. Si hubiera mas de uno, devuelve el de menor placa.]

  \InterfazFuncion{ConMismasSanciones}{\In{a}{placa}, \In{as}{agentes}}{conj(placa)}%
  [estaAgente($a, as$)]
  {alias($res \igobs$ conMismasSanciones($a, as$))}%
  [$\Theta(1)$]
  [Devuelve la referencia al conjunto de agentes que tienen la misma cantidad de sanciones que el agente $a$]
  [$res$ no es modificable. Cualquier referencia que se guarde queda invalidada si se agregan sanciones.]

  \InterfazFuncion{ConKSanciones}{\In{k}{nat}, \Inout{as}{agentes}}{conj(placa)}%
  [true]
  {alias($res \igobs$ conKSanciones($k, as$))}%
  [O(N$a$) la primera vez, O(log(N$a$)) en siguientes llamadas mientras no ocurran sanciones.]
  [Devuelve el conjunto de agentes que tienen exactamente $k$ sanciones. N$a$ es la cantidad de agentes definidos en $as$]
  [$res$ no es modificable. Cualquier referencia que se guarde queda invalidada si se agregan sanciones. Si $res$ es $NULL$, significa que no se encontraron agentes con $k$ sanciones]


\end{Interfaz}

\begin{Representacion}
  
  \Titulo{Representacion de los Agentes}

  \begin{Estructura}{agentes}[estr]
    \begin{Tupla}[estr]
      \tupItem{as}{DiccRapido(placa, datos)}%
      \tupItem{claves}{conj(placa)}
      \tupItem{masVig}{placa}%
      \tupItem{huboSanciones}{bool}%
      \tupItem{mismSanciones}{lista(conj(placa))}%
      \tupItem{kSanciones}{Arreglo(tuplaK)}%
    \end{Tupla}

    \begin{Tupla}[datos]
      \tupItem{pos}{posicion}%
      \tupItem{sanciones}{nat}%
      \tupItem{capturas}{nat}%
      \tupItem{conMismSanciones}{itLista(conj(placa))}%
      \tupItem{itConjMismSanciones}{itConj(placa)}
    \end{Tupla}

    \begin{Tupla}[tuplaK]
      \tupItem{sanciones}{nat}%
      \tupItem{placa}{placa}%
    \end{Tupla}

    \begin{Tupla}[posicion]
      \tupItem{fila}{nat}%
      \tupItem{columna}{nat}%
    \end{Tupla}

    {\small La idea de la lista enlazada mismSanciones es que guarde en cada posicion a todos aquellos agentes que comparten sanciones, con rapido acceso gracias al Iterador en los datos del agente. El arreglo kSanciones se utiliza para ordenar a los agentes por su cantidad de sanciones en tiempo O(N), y poder buscar a alguno con $K$ sanciones en O(log(N)) para acceder a aquellos que tienen la misma cantidad via mismSanciones}


  \end{Estructura}

  Invariante de representacion en castellano:
  \begin{enumerate}
  
  \item $claves$ son las claves del diccionario $as$
  \item masVigilante esta definido en agentes.
  \item masVigilante es el agente con menor numero de placa entre aquellos que tienen mas capturas en el diccionario de agentes.
  \item El arreglo kSanciones tiene almacenadas todas las placas de agentes.
  \item Si no hubo sanciones ($\neg$huboSanciones), entonces el arreglo kSanciones representa a los agentes en orden creciente de sanciones. Ademas las sanciones se corresponden con el diccionario
  \item Los agentes de la lista mismSanciones no estan repetidos, y son exactamente los definidos en diccionario de agentes.
  %\item No se repiten los agentes en conjuntos de distintos items de la lista.
  \item Para todo item de la lista mismSanciones, y para todo agente dentro del conjunto del item, la cantidad de sanciones es igual al resto del conjunto, y menor al de todos los agentes de items siguientes.
  \item Para todo agente, su conMismSanciones apunta al item de la lista donde esta el conjunto que lo incluye, y ademas itConjMismSanciones apunta al mismo agente.
  
  \end{enumerate}

  \Rep[estr][e]{
  
  \begin{enumerate}
  \item ( ($\forall$ $a$ : nat) def?($a$, e.as) $\iff$ Pertenece?(e.claves, $a$) ) $\land$
  \item def?(e.masVigilante, e.as) $\yluego$
  \item (($\forall$ $a$ : nat) def?($a$, e.as) $\impluego$ (obtener($a$, e.as).capturas $=$ obtener(e.masVigilante, e.as).capturas $\land$ $a$ $>$ e.masVigilante) $\lor$ (obtener($a$, e.as).capturas $<$ obtener(e.masVigilante, e.as).capturas) $\lor$ ($a$ $=$ e.masVigilante)) $\land$
  \item ( ($\forall$ $a$ : nat) def?($a$, e.as) $\impluego$ ($\exists!$ $i$ : nat) e.kSanciones[$i$].placa $=$ $a$  $\land$
  \item $\neg$(e.huboSanciones) $\implies$ (CorrespondenSanciones(e) $\yluego$ SancionesOrdenadas(e)) $\land$
  \item ( \ (($\forall$ $i$ : nat) $i <$ Longitud(e.mismSanciones) $\impluego$ MSDefinidosEnDicc(e,$i$)) $\land$ DiccDefinidosEnMSYNoRepetidos(e) \ ) $\yluego$
  \item ( \ ($\forall$ $i$ : nat) ($i <$ Longitud(e.mismSanciones) $\impluego$ MSTieneMismSanciones(e, $i$)) $\land$ \\ ($i <$ (Longitud(e.mismSanciones) $- 1$) $\impluego$ MSCadaItemTieneDifSanciones(e,$i$)) $\land$
  \item ( ($\forall$ $a$ : nat) def?($a$, e.as) $\impluego$ ( \ Pertenece?(Siguiente(obtener($a$, e.as).conMismSanciones), $a$) $\land$ (Siguiente(obtener($a$, e.as).itConjMismSanciones) $=$ $a$) \ ) ) \ )
  \end{enumerate}
  
  }\mbox{}
  {Reemplazos sintacticos: }
  ~      
  % \tadOperacion{CorrespondenSanciones}{estr/e}{bool}{}
  \tadAxioma{CorrespondenSanciones($e$)}{($\forall$ $i$ : nat) $i$ $<$ Tam(e.kSanciones) $\impluego$ e.kSanciones[$i$].sanciones $=$ Obtener(e.kSanciones[$i$].placa, e.as).sanciones}
  ~      
  % \tadOperacion{SancionesOrdenadas}{estr/e}{bool}{}
  \tadAxioma{SancionesOrdenadas($e$)}{($\forall$ $i$ : nat) $i$ $<$ (Tam(e.kSanciones)$-1$) $\impluego$ Obtener(e.kSanciones[$i$].placa, e.as).sanciones $\leq$ Obtener(e.kSanciones[$i+1$].placa, e.as).sanciones}
  ~      
  % \tadOperacion{MSDefinidosEnDicc}{estr/e, nat/i}{bool}{}
  \tadAxioma{MSDefinidosEnDicc($e$, $i$)}{($\forall$ $a$ : nat) Pertenece?(e.mismSanciones[$i$], $a$) $\implies$ Def?($a$, e.as)}
  ~      
  %\tadOperacion{DiccDefinidosEnMSYNoRepetidos}{estr/e}{bool}{}
  \tadAxioma{DiccDefinidosEnMSYNoRepetidos($e$)}{($\forall$ $a$ : nat) Def?($a$, e.as) $\impluego$ (($\exists!$ $i$ : nat) i $<$ Longitud(e.mismSanciones) $\yluego$ Pertenece?(e.mismSanciones[$i$], $a$))}
  ~      
  %\tadOperacion{MSTieneMismSanciones}{estr/e, nat/i}{bool}{}
  \tadAxioma{MSTieneMismSanciones($e$, $i$)}{($\forall$ $a, a'$ : nat) (Pertenece?(e.mismSanciones[$i$], $a$) $\land$ Pertenece?(e.mismSanciones[$i$], $a'$) $\land$ $\neg$($a = a'$)) $\impluego$ Obtener($a$, e.as).sanciones $=$ Obtener($a'$, e.as).sanciones}
  ~      
  %\tadOperacion{MSCadaItemTieneDifSanciones}{estr/e, nat/i}{bool}{}
  \tadAxioma{MSCadaItemTieneDifSanciones($e$, $i$)}{($\forall$ $a, a'$ : nat) (Pertenece?(e.mismSanciones[$i$], $a$) $\land$ Pertenece?(e.mismSanciones[$i+1$], $a'$)) $\impluego$ Obtener($a$, e.as).sanciones $<$ Obtener($a'$, e.as).sanciones}
  ~
  
  \AbsFc[estr]{agentes}[e]{$as$ : agentes / \\
    ($\forall$ $a$ : placa) Def?($a$, e.as) $\iff$ $a$ $\in$ agentes?($as$) $\yluego$ \\
    ($\forall$ $a$ : placa) Def?($a$, e.as) $\impluego$ \\
    ( Obtener($a$, e.as).sanciones $\igobs$ sancionesAgente($a$, $as$) $\land$ \\
    Obtener($a$, e.as).capturas $\igobs$ capturasAgente($a$, $as$) $\land$ \\
    Obtener($a$, e.as).pos $\igobs$ posicionAgente($a$, $as$) $\land$ )
    }

\end{Representacion}

  \begin{Algoritmos}

\textbf{Algoritmos de Agentes}
  \listofalgorithms
  \bigskip

  \begin{algorithm}[H]
    \text{$i$NuevoAgentes(\In{d}{dicc(placa,posicion)}) $\rightarrow$ res: estr}
  
    \Begin{
      var \\
      itAMismSanciones : itConj \\
      iteradorDicc : itDicc \\
      i : nat \\
      i $\leftarrow$ 0 \hfill \textbf{//O(1)} \\
      res.as $\leftarrow$ DiccRapidoVacio() \hfill \textbf{//O(1)} \\
      % res.claves $\leftarrow$ Claves(d) \hfill \textbf{//O(N)} \\
      res.claves $\leftarrow$ Vacio() \hfill \textbf{//O(1)} \\
      iteradorDicc $\leftarrow$ CrearIt(d) \hfill \textbf{//O(1)} \\

      \While(\hfill \textbf{//O(1)}){HaySiguiente(iteradorDicc)}
        {
          res.claves.Agregar(SiguienteClave(iteradorDicc)) \hfill \textbf{//O(1)} \\
          Avanzar(iteradorDicc) \hfill \textbf{//O(1)} \\
        }
        \hfill \textbf{//While: O(N$a$)}

      res.masVig $\leftarrow$ 0 \hfill \textbf{//O(1)} \\
      res.mismSanciones $\leftarrow$ Vacia() \hfill \textbf{//O(1)} \\
      AgregarAtras(res.mismSanciones, res.claves) \hfill \textbf{//O(N$a$)} \\
      itAMismSanciones $\leftarrow$ CrearIt(res.mismSanciones) \hfill \textbf{//O(1)} \\
      res.kSanciones $\leftarrow$ CrearArreglo(Longitud(res.claves)) \hfill \textbf{//O(N$a$)}

      % itClavesD $\leftarrow$ CrearIt(res.claves) \hfill \textbf{//O(1)} \\
      iteradorDicc $\leftarrow$ CrearIt(d) \hfill \textbf{//O(1)} \\
      \While(\hfill \textbf{//Guarda: O(1)}){HaySiguiente(iteradorDicc)}{
        \If(\hfill \textbf{//O(1)}){res.masVig == 0}
          {res.masVig $\leftarrow$ SiguienteClave(iteradorDicc) \hfill \textbf{//O(1)}}
        \If(\hfill \textbf{//O(1)}){SiguienteClave(iteradorDicc) $<$ res.masVig}
          {res.masVig $\leftarrow$ SiguienteClave(iteradorDicc) \hfill \textbf{//O(1)}}
        Definir (  SiguienteClave(iteradorDicc),  $\langle$ SiguienteSignificado(iteradorDicc), 0, 0, itAMismSanciones $\rangle$,  res.as ) \hfill \textbf{//O(1)} \\
        res.kSanciones[i].placa $\leftarrow$ SiguienteClave(iteradorDicc) \hfill \textbf{//O(1)} \\
        res.kSanciones[i].sanciones $\leftarrow$ 0 \hfill \textbf{//O(1)} \\
        i $\leftarrow$ i+1 \hfill \textbf{//O(1)} \\
        Avanzar(iteradorDicc) \hfill \textbf{//O(1)} 
      }

      \hfill \textbf{//While: O(N$a$)} \\
      res.huboSanciones $\leftarrow$ false \hfill \textbf{//O(1)} \\
    }
    \KwData{O(N$a$), con N$a$ la cantidad de agentes}
  
    \caption{NuevoAgentes}
  \end{algorithm}
  \rule{17.5cm}{0.4pt}
  \bigskip

  \begin{algorithm}[H]
    \text{$i$Agentes?(\In{as}{estr}) $\rightarrow$ res: itConj(nat)}
  
    \Begin{
      res $\leftarrow$ CrearIt(as.claves) \hfill \textbf{//O(1)}
    }
    \KwData{O(1)}
  
    \caption{Agentes?}
  \end{algorithm}
  \rule{17.5cm}{0.4pt}
  \bigskip

  \begin{algorithm}[H]
    \text{$i$AgregarSancion( \In{a}{nat}, \Inout{as}{estr} ) }
  
    \Begin{
      var \\
      iteradorLista : itLista \\
      iteradorConjunto : itConj \\
      {\small //Primero, le agrego la sancion directamente sobre el significado (DiccRapido hace aliasing en la operacion Obtener) \\ }
      *(Obtener(a, as.as)).sanciones $\leftarrow$ (*(Obtener(a, as.as)).sanciones + 1) \hfill \textbf{//O(N$a$)} \\
      as.huboSanciones $\leftarrow$ true \hfill \textbf{//O(1)} \\
      {\small //Ahora tengo que modificar mismSanciones para que refleje el cambio \\ }
      iteradorLista $\leftarrow$ *(Obtener(a, as.as)).conMismSanciones \hfill \textbf{//O(N$a$)} \\
      iteradorConjunto $\leftarrow$ *(Obtener(a, as.as)).itConjMismSanciones \hfill \textbf{//O(N$a$)} \\
      {\small //Borro a $a$ del conjunto, porque ya no comparte sanciones con nadie del mismo \\}
      % Borrar(a, Siguiente(iteradorLista)) \hfill \textbf{//O(N), pero se desestima} \\
      EliminarSiguiente(iteradorConjunto) \hfill \textbf{//O(1)} \\

      \eIf(\hfill \textbf{//O(1)}){EsVacio?(Siguiente(iteradorLista))}
        {
          {\small //El conjunto en el que estaba quedo vacio, entonces lo borro de la lista \\}
          EliminarSiguiente(iteradorLista) \hfill \textbf{//O(1)} \\
        }
        {
          {\small //Me muevo al lugar que le corresponde ahora \\}
          Avanzar(iteradorLista) \hfill \textbf{//O(1)} \\
        } 
      
      \eIf(\hfill \textbf{//O(1)}){$\neg$ HaySiguiente(iteradorLista)}
        {
          {\small //Si no hay nada, creo un nuevo elemento que solo me tiene a mi \\}
          AgregarComoSiguiente(iteradorLista, Vacio()) \hfill \textbf{//O(1)} \\
          iteradorConjunto $\leftarrow$ Siguiente(iteradorLista).Agregar(a) \hfill \textbf{//O(1)} \\
        }
        {
        \eIf(\hfill \textbf{//O(1)}){*(Obtener(DameUno(Siguiente(iteradorLista)), as)).sanciones $>$ *(Obtener(a, as.as)).sanciones}
          {
          {\small //Si el que esta en el lugar al que iba tiene mas sanciones que yo, me agrego antes para mantener el orden creciente \\}
          AgregarComoAnterior(iteradorLista, Vacio()) \hfill \textbf{//O(1)} \\
          iteradorConjunto $\leftarrow$ Anterior(iteradorLista).Agregar(a) \hfill \textbf{//O(1)} \\
          Retroceder(iteradorLista) \hfill \textbf{//O(1)} \\
          }
          {
          {\small //Si no, debe tener las mismas (tiene mas que las que yo tenia antes, y yo sume una sancion, a lo sumo tiene la misma cantidad) \\}
          iteradorConjunto $\leftarrow$ Siguiente(iteradorLista).Agregar(a) \hfill \textbf{//O(1)}
          } 
        }
      *(Obtener(a, as.as)).conMismSanciones $\leftarrow$ iteradorLista \hfill \textbf{//O(N$a$)} \\
      *(Obtener(a, as.as)).itConjMismSanciones $\leftarrow$ iteradorConjunto \hfill \textbf{//O(N$a$)} \\
    }
    \KwData{O(N$a$) en peor caso. O(1) si se asegura distribucion uniforme de las placas}
  
    \caption{AgregarSancion}
  \end{algorithm}
  \rule{17.5cm}{0.4pt}
  \bigskip

  \begin{algorithm}[H]
    \text{$i$CambiarPosicion(\In{a}{nat}, \In{p}{posicion}, \Inout{as}{estr})}
  
    \Begin{
      *(Obtener(a, as.as)).posicion $\leftarrow$ p \hfill \textbf{//O(N$a$)} \\
    }
    \KwData{O(N$a$) en peor caso. O(1) si se asegura distribucion uniformede las placas}
  
    \caption{CambiarPosicion}
  \end{algorithm}
  \rule{17.5cm}{0.4pt}
  \bigskip

  \begin{algorithm}[H]
    \text{$i$AgregarCaptura(\In{a}{nat}, \Inout{as}{estr})}
  
    \Begin{
      *(Obtener(a, as.as)).capturas $\leftarrow$ (*(Obtener(a, as.as)).capturas + 1) \hfill \textbf{//O(N$a$)} \\
      {\small //Ademas de sumar captura, hago el mantenimiento de masVigilante  \\}
      \eIf(\hfill \textbf{//O(N$a$)}){*(Obtener(a, as.as)).capturas $>$ *Obtener(as.masVigilante, as).capturas}
        {as.masVigilante $\leftarrow$ a \hfill \textbf{//O(1)}}
        {
          \If(\hfill \textbf{//O(N$a$)}){*(Obtener(a, as.as)).capturas $==$ *Obtener(as.masVigilante, as).capturas}
          {
            \If(\hfill \textbf{//O(1)}){a $<$ as.masVigilante}
            {as.masVigilante $\leftarrow$ a \hfill \textbf{//O(1)}}
          }
        }
    }
    \KwData{O(N$a$) en peor caso. O(1) si se asegura distribucion uniforme de las placas}
  
    \caption{AgregarCaptura}
  \end{algorithm}
  \rule{17.5cm}{0.4pt}
  \bigskip

  \begin{algorithm}[H]
    \text{$i$PosAgente(\In{a}{nat}, \In{as}{estr}) $\rightarrow$ res: posicion}
  
    \Begin{
      res $\leftarrow$ *(Obtener(a, as.as)).posicion \hfill \textbf{//O(N$a$)}
    }
    \KwData{O(N$a$) en peor caso. O(1) si se asegura distribucion uniforme de las placas}
  
    \caption{PosAgente}
  \end{algorithm}
  \rule{17.5cm}{0.4pt}
  \bigskip

  \begin{algorithm}[H]
    \text{$i$SancionesAgente(\In{a}{nat}, \In{as}{estr}) $\rightarrow$ res: nat}
  
    \Begin{
      res $\leftarrow$ *(Obtener(a, as.as)).sanciones \hfill \textbf{//O(N$a$)}
    }
    \KwData{O(N$a$) en peor caso. O(1) si se asegura distribucion uniforme de las placas}
  
    \caption{SancionesAgente}
  \end{algorithm}
  \rule{17.5cm}{0.4pt}
  \bigskip

  \begin{algorithm}[H]
    \text{$i$CapturasAgente(\In{a}{nat}, \In{as}{estr}) $\rightarrow$ res: nat}
  
    \Begin{
      res $\leftarrow$ *(Obtener(a, as.as)).capturas \hfill \textbf{//O(N$a$)}
    }
    \KwData{O(N$a$) en peor caso. O(1) si se asegura distribucion uniforme de las placas}
  
    \caption{CapturasAgente}
  \end{algorithm}
  \rule{17.5cm}{0.4pt}
  \bigskip

  \begin{algorithm}[H]
    \text{$i$masVigilante(\In{as}{estr}) $\rightarrow$ res: nat}
  
    \Begin{
      res $\leftarrow$ as.masVigilante \hfill \textbf{//O(1)}
    }
    \KwData{O(1)}
  
    \caption{masVigilante}
  \end{algorithm}
  \rule{17.5cm}{0.4pt}
  \bigskip

  \begin{algorithm}[H]
    \text{$i$ConMismasSanciones(\In{a}{nat}, \In{as}{estr}) $\rightarrow$ res: conj(nat)}
  
    \Begin{
      res $\leftarrow$ Siguiente(*(Obtener(a, as.as)).conMismSanciones) \hfill \textbf{//O(N$a$)}
    }
    \KwData{O(N$a$) en peor caso. O(1) si se asegura distribucion uniforme de las placas}

  
    \caption{ConMismasSanciones}
  \end{algorithm}
  \rule{17.5cm}{0.4pt}
  \bigskip

  \begin{algorithm}[H]
    \text{$i$ConKSanciones(\In{k}{nat}, \Inout{as}{estr}) $\rightarrow$ res: conj(nat)}
  
    \Begin{
      var \\
      i : nat \\
      encontrado : bool \\
      iteradorLista : itLista(conj(placa))\\
      iteradorConjunto : itConj(placa) \\
      encontrado $\leftarrow$ false \hfill \textbf{//O(1)} \\
      i $\leftarrow$ 0 \hfill \textbf{//O(1)} \\
      \If(\hfill \textbf{//O(1)}){as.huboSanciones}
        {
          {\small //Actualizo el arreglo kSanciones  \\}
          % \While(\hfill \textbf{//O(1)}){i $<$ Tam(as.kSanciones)}
          % {
          %   as.kSanciones[i].sanciones $\leftarrow$ *Obtener(as.kSanciones[i].placa, as.as).sanciones \hfill \textbf{//O(1)} \\
          %   i $\leftarrow$ i + 1 \hfill \textbf{//O(1)}
          % }
          % \hfill \textbf{//While: O(N)} \\
          % {\small //Y luego lo ordeno en orden creciente de sanciones  \\}
          % CountingSortSanciones(as.kSanciones, as) \hfill \textbf{//O(N)} \\

          iteradorLista $\leftarrow$ CrearIt(as.mismSanciones) \hfill \textbf{//O(1)} \\
          \While(\hfill \textbf{//O(1)}){HaySiguiente(iteradorLista)}
            {
              iteradorConjunto $\leftarrow$ CrearIt(Siguiente(iteradorLista)) \hfill \textbf{//O(1)} \\
              \While(\hfill \textbf{//O(1)}){HaySiguiente(iteradorConjunto)}
                {
                  as.kSanciones[i].placa $\leftarrow$ Siguiente(iteradorConjunto) \hfill \textbf{//O(1)} \\
                  as.kSanciones[i].sanciones $\leftarrow$ *(Obtener(Siguiente(iteradorConjunto), as.as)).sanciones \hfill \textbf{//O(N$a$)} \\
                  i $\leftarrow$ i + 1 \hfill \textbf{//O(1)} \\
                }
            }
            {\small //Por el punto 6 y 7 del Rep, recorrer todos los conjuntos de cada item de la lista mismSanciones en orden es equivalente a recorrer a cada uno de los agentes en orden creciente de sanciones. } \\
            \hfill \textbf{//While (En el peor caso se recorren N$a$ agentes por el punto 6 del Rep): O(N$a$ * N$a$)}

        }
        
        {\small //En el arreglo kSanciones tengo, en orden creciente por sancion, a los agentes. Busco a uno con k sanciones  \\}
        {\small //Busqueda binaria me devuelve el i que cumple que las sanciones del agente en posicion i del arreglo son k   \\}
        encontrado $\leftarrow$ BusquedaBinariaSanciones(k, as.kSanciones, as, i) \hfill \textbf{//O(log(N$a$))} \\
        \eIf(\hfill \textbf{//O(1)}){encontrado}
          {
            res $\leftarrow$ Siguiente(*Obtener(as.kSanciones[i], as.as).conMismSanciones) \hfill \textbf{//O(N$a$)} \\
          }
          {
            res $\leftarrow$ Vacio() \hfill \textbf{//O(1)} \\
          }
        
        
    }
    \KwData{Peor caso: O(N$a^{2}$ + N$a$ + log(N$a$)) = O(N$a^{2}$) |  En llamadas siguientes: O(N$a$ + log(N$a$))}
    \KwData{Caso promedio si hay distribucion uniforme de placas: O(N$a$ + log(N$a$)) = O(N$a$) |  En llamadas siguientes: O(log(N$a$))}
    \KwComm{La primer complejidad se da cuando hubo sanciones. N$a$ es la cantidad de agentes.}
    \caption{ConKSanciones}
  \end{algorithm}
  \rule{17.5cm}{0.4pt}
  \bigskip

  % \begin{algorithm}[H]
  %   \text{CountingSortSanciones(\Inout{arreglo}{Arreglo(tuplaK)}, \In{as}{as})}
  
  %   \Begin{
  %     var \\
  %     i, total, cantidadAnt : nat \\
  %     maxSanciones : nat \\
  %     cantidad : Arreglo(nat) \\
  %     output : Arreglo(tuplaK) \\

  %     iterador : itConj(nat)\\
  %     i $\leftarrow$ 0 \hfill \textbf{//O(1)} \\
  %     cantidadAnt $\leftarrow$ 0 \hfill \textbf{//O(1)} \\
  %     total $\leftarrow$ 0 \hfill \textbf{//O(1)} \\
  %     {\small //Se que a todos los agentes con las maximas sanciones los tengo en el ultimo elemento de la lista mismSanciones.  \\}
  %     maxSanciones $\leftarrow$ *Obtener(DameUno(Ultimo(as.mismSanciones)), as.as).sanciones \hfill \textbf{//O(1)} \\
  %     {\small //Creo el arreglo cantidad con MaxSanciones posiciones. \\}
  %     cantidad $\leftarrow$ CrearArreglo(maxSanciones) \hfill \textbf{//O(maxSanciones)} \\
  %     output $\leftarrow$ CrearArreglo(Tam(arreglo)) \hfill \textbf{//O(N)} \\

  %     \While(\hfill \textbf{//O(1)}){i $<$ maxSanciones}
  %     {
  %       cantidad[i] $\leftarrow$ 0 \hfill \textbf{//O(1)} \\
  %       i $\leftarrow$ i + 1 \hfill \textbf{//O(1)} 
  %     }
  %     \hfill \textbf{//While: O(maxSanciones)} \\


  %     {\small //Calculo la cantidad de cada numero de sanciones  \\}
  %     i $\leftarrow$ 0 \hfill \textbf{//O(1)} \\
  %     \While(\hfill \textbf{//O(1)}){i $<$ Tam(arreglo)}
  %     {
  %       cantidad[arreglo[i].sanciones] $\leftarrow$ cantidad[arreglo[i].sanciones] + 1 \hfill \textbf{//O(1)} \\
  %       i $\leftarrow$ i + 1 \hfill \textbf{//O(1)}
  %     }
  %     \hfill \textbf{//While: O(N)}

  %     {\small //Calculo el indice inicial para cada numero de sanciones  \\}
  %     i $\leftarrow$ 0 \hfill \textbf{//O(1)} \\
  %     \While(\hfill \textbf{//O(1)}){i $<$ maxSanciones}
  %     {
  %       cantidadAnt $\leftarrow$ cantidad[i] \hfill \textbf{//O(1)} \\
  %       cantidad[i] $\leftarrow$ total \hfill \textbf{//O(1)} \\
  %       total $\leftarrow$ total + cantidadAnt \hfill \textbf{//O(1)} \\
  %       i $\leftarrow$ i + 1 \hfill \textbf{//O(1)} 
  %     }
  %     \hfill \textbf{//While: O(maxSanciones)} \\

  %     {\small //Coloco a cada agente (<sanciones, placa>) en el lugar que le corresponde en el output  \\}
  %     i $\leftarrow$ 0 \hfill \textbf{//O(1)} \\
  %     \While(\hfill \textbf{//O(1)}){i $<$ Tam(arreglo)}
  %     {
  %       output[cantidad[arreglo[i].sanciones]] $\leftarrow$ arreglo[i] \hfill \textbf{//O(1)} \\
  %       cantidad[arreglo[i].sanciones] $\leftarrow$ cantidad[arreglo[i].sanciones] + 1 \hfill \textbf{//O(1)} \\
  %       i $\leftarrow$ i + 1 \hfill \textbf{//O(1)}
  %     }
  %     \hfill \textbf{//While: O(N)}

  %     arreglo $\leftarrow$ output \hfill \textbf{//O(N)}
  %   }
  %   \KwData{O(5N + 3maxSanciones) = O(N)}
  %   \KwComm{Vamos a asumir que maxSanciones es, a lo sumo, un multiplo de N (kN) con un k constante y peque�o (k $<$ N), y tomar la complejidad como O(N).}
  
  %   \caption{CountingSortSanciones}
  % \end{algorithm}
  % \rule{17.5cm}{0.4pt}
  % \bigskip

  \begin{algorithm}[H]
    \text{BusquedaBinariaSanciones(\In{k}{nat}, \In{arreglo}{Arreglo(tuplaK)}, \In{as}{estr} \Inout{i}{nat}) $\rightarrow$ res: bool}
  
    \Begin{
      var \\
      iMin, iMax, iMed : nat \\
      iMin $\leftarrow$ 0 \hfill \textbf{//O(1)} \\
      iMax $\leftarrow$ Tam(arreglo) \hfill \textbf{//O(1)} \\
      res $\leftarrow$ false \hfill \textbf{//O(1)} \\

      \While(\hfill \textbf{//O(1)}){iMin $\leq$ iMax}
      {
        iMed $\leftarrow$ $\lfloor((iMin + iMax) / 2)\rfloor$ \hfill \textbf{//O(1)} \\
        \eIf(\hfill \textbf{//O(1)}){arreglo[iMed].sanciones == k}
          {
            i $\leftarrow$ iMed \hfill \textbf{//O(1)} \\
            res $\leftarrow$ true \hfill \textbf{//O(1)} \\
            iMin $\leftarrow$ $iMax + 1$ \hfill \textbf{//O(1)}
          }
          {
            \eIf(\hfill \textbf{//O(1)}){arreglo[iMed].sanciones $<$ k}
              {
                iMin $\leftarrow$ $iMed + 1$ \hfill \textbf{//O(1)}
              }
              {
                iMax $\leftarrow$ $iMed - 1$ \hfill \textbf{//O(1)}
              }
          }

      }
      \hfill \textbf{//While: O(log(N))} \\
      {\small //La complejidad es log(N), ya que en la primer iteracion iMin = 0 y iMax = Tam(Arreglo), \\ // y despues de cada iteracion una de las variables toma el indice intermedio, por lo que el rango pasa de \\ //  $(iMin, iMax)$ a ($\frac{iMin+iMax}{2}$, $iMax$) o ($iMin$, $\frac{iMin+iMax}{2}$), cuyas longitudes son la mitad que la del rango anterior. \\ //La cantidad de veces que se puede dividir al arreglo y mirar solo una mitad es igual a log$_{2}$(N) \\}

    }
    \KwData{O(log(N))}
    \KwComm{$i$ devuelve el indice donde el agente tiene $k$ sanciones. $i$ se invalida si $res$ es $false$ (no encontrado)}

  
    \caption{BusquedaBinariaSanciones}
  \end{algorithm}
  \rule{17.5cm}{0.4pt}
  \bigskip

  \end{Algoritmos}


\end{document}
