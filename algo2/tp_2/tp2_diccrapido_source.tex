\documentclass[a4paper,10pt]{article}
\usepackage[paper=a4paper, hmargin=1.5cm, bottom=1.5cm, top=3.5cm]{geometry}
\usepackage[latin1]{inputenc}
\usepackage[T1]{fontenc}
\usepackage[spanish]{babel}
\usepackage{xspace}
\usepackage{xargs}
\usepackage{ifthen}
\usepackage{aed2-tad,aed2-symb,aed2-itef}
\usepackage[]{algorithm2e}

\SetKwInput{KwData}{Complejidad}
\SetKwInput{KwComm}{Comentarios}
\SetKwInput{KwExp}{Explicacion}
\SetAlgorithmName{Algoritmo}{Algoritmo}{Lista de algoritmos}
\SetAlCapSkip{1em}


\newcommand{\moduloNombre}[1]{\textbf{#1}}

\let\NombreFuncion=\textsc
\let\TipoVariable=\texttt
\let\ModificadorArgumento=\textbf
\newcommand{\res}{$res$\xspace}
\newcommand{\tab}{\hspace*{7mm}}

\newcommandx{\TipoFuncion}[3]{%
  \NombreFuncion{#1}(#2) \ifx#3\empty\else $\to$ \res\,: \TipoVariable{#3}\fi%
}
\newcommand{\In}[2]{\ModificadorArgumento{in} \ensuremath{#1}\,: \TipoVariable{#2}\xspace}
\newcommand{\Out}[2]{\ModificadorArgumento{out} \ensuremath{#1}\,: \TipoVariable{#2}\xspace}
\newcommand{\Inout}[2]{\ModificadorArgumento{in/out} \ensuremath{#1}\,: \TipoVariable{#2}\xspace}
\newcommand{\Aplicar}[2]{\NombreFuncion{#1}(#2)}

\newlength{\IntFuncionLengthA}
\newlength{\IntFuncionLengthB}
\newlength{\IntFuncionLengthC}
%InterfazFuncion(nombre, argumentos, valor retorno, precondicion, postcondicion, complejidad, descripcion, aliasing)
\newcommandx{\InterfazFuncion}[9][4=true,6,7,8,9]{%
  \hangindent=\parindent
  \TipoFuncion{#1}{#2}{#3}\\%
  \textbf{Pre} $\equiv$ \{#4\}\\%
  \textbf{Post} $\equiv$ \{#5\}%
  \ifx#6\empty\else\\\textbf{Complejidad:} #6\fi%
  \ifx#7\empty\else\\\textbf{Descripci�n:} #7\fi%
  \ifx#8\empty\else\\\textbf{Aliasing:} #8\fi%
  \ifx#9\empty\else\\\textbf{Requiere:} #9\fi%
}

\newenvironment{Interfaz}{%
  \parskip=2ex%
  \noindent\textbf{\Large Interfaz}%
  \par%
}{}

\newenvironment{Representacion}{%
  \vspace*{2ex}%
  \noindent\textbf{\Large Representaci�n}%
  \vspace*{2ex}%
}{}

\newenvironment{Algoritmos}{%
  \vspace*{2ex}%
  \noindent\textbf{\Large Algoritmos}%
  \vspace*{2ex}%
}{}


\newcommand{\Titulo}[1]{
  \vspace*{1ex}\par\noindent\textbf{\large #1}\par
}

\newenvironmentx{Estructura}[2][2={estr}]{%
  \par\vspace*{2ex}%
  \TipoVariable{#1} \textbf{se representa con} \TipoVariable{#2}%
  \par\vspace*{1ex}%
}{%
  \par\vspace*{2ex}%
}%

\newboolean{EstructuraHayItems}
\newlength{\lenTupla}
\newenvironmentx{Tupla}[1][1={estr}]{%
    \settowidth{\lenTupla}{\hspace*{3mm}donde \TipoVariable{#1} es \TipoVariable{tupla}$($}%
    \addtolength{\lenTupla}{\parindent}%
    \hspace*{3mm}donde \TipoVariable{#1} es \TipoVariable{tupla}$($%
    \begin{minipage}[t]{\linewidth-\lenTupla}%
    \setboolean{EstructuraHayItems}{false}%
}{%
    $)$%
    \end{minipage}
}

\newcommandx{\tupItem}[3][1={\ }]{%
    %\hspace*{3mm}%
    \ifthenelse{\boolean{EstructuraHayItems}}{%
        ,#1%
    }{}%
    \emph{#2}: \TipoVariable{#3}%
    \setboolean{EstructuraHayItems}{true}%
}

\newcommandx{\RepFc}[3][1={estr},2={e}]{%
  \tadOperacion{Rep}{#1}{bool}{}%
  \tadAxioma{Rep($#2$)}{#3}%
}%

\newcommandx{\Rep}[3][1={estr},2={e}]{%
  \tadOperacion{Rep}{#1}{bool}{}%
  \tadAxioma{Rep($#2$)}{true \ssi #3}%
}%

\newcommandx{\Abs}[5][1={estr},3={e}]{%
  \tadOperacion{Abs}{#1/#3}{#2}{Rep($#3$)}%
  \settominwidth{\hangindent}{Abs($#3$) \igobs #4: #2 $\mid$ }%
  \addtolength{\hangindent}{\parindent}%
  Abs($#3$) \igobs #4: #2 $\mid$ #5%
}%

\newcommandx{\AbsFc}[4][1={estr},3={e}]{%
  \tadOperacion{Abs}{#1/#3}{#2}{Rep($#3$)}%
  \tadAxioma{Abs($#3$)}{#4}%
}%


\newcommand{\DRef}{\ensuremath{\rightarrow}}

\begin{document}


\section{Modulo DiccRapido(clave, significado)}


\begin{Interfaz}
  
  \textbf{se explica con}: \tadNombre{Diccionario(clave, significado)}

  \textbf{generos}: \TipoVariable{diccRapido(nat, significado)}

  \Titulo{Operaciones basicas de conjunto}

  \InterfazFuncion{Vacio}{\In{maxClaves}{nat}}{diccRapido(nat, significado)}
  [true]
  {$res \igobs$ vacio()}
  [O(N) donde N es $maxClaves$]
  [Crea un nuevo diccionario vacio.]

  \InterfazFuncion{Definir}{\In{k}{clave}\In{s}{significado}, \Inout{d}{diccRapido(nat, significado)}}{}%
  [d $\igobs$ $d_{0}$]
  {$d \igobs$ definir($k$, $s$, $d_{0}$)}%
  [O(N) en peor caso. O(1) en el caso promedio si se asegura distribucion uniforme de las claves.]
  [Define el elemento $k$, con significado $s$, en el diccionario $d$]

  \InterfazFuncion{Def?}{\In{c}{clave}, \In{d}{diccRapido(nat, significado)}}{bool}%
  [true]
  {$res \igobs$ def?($c$, $d$)}%
  [O(N) en peor caso. O(1) en el caso promedio si se asegura distribucion uniforme de las claves.]
  [Devuelve $true$ si la clave $k$ esta definida]

  \InterfazFuncion{Obtener}{\In{c}{clave}, \In{d}{diccRapido(nat, significado)}}{significado}%
  [def?($c$, $d$)]
  {alias($res \igobs$ obtener($c$, $d$))}%
  [O(N) en peor caso. O(1) en el caso promedio si se asegura distribucion uniforme de las claves.]
  [Devuelve el significado de la clave $c$]
  [$res$ es una referencia al significado de $c$ en el diccionario $d$. Si se modifica, se modificara el significado dentro del diccionario. Si se borra una clave, o se define alguna clave, la referencia queda invalidada.]

  \InterfazFuncion{Borrar}{\In{c}{clave}, \Inout{d}{diccRapido(nat, significado)}}{}%
  [def?($c$, $d$)]
  {$res \igobs$ borrar($c$, $d$)}%
  [O(N) en peor caso. O(1) en el caso promedio si se asegura distribucion uniforme de las claves.]
  [Devuelve el significado de la clave $c$]
  [$res$ es una referencia al significado de $c$ en el diccionario $d$. Si se modifica, se modificara el significado dentro del diccionario. Si se borra una clave, o se define alguna clave, la referencia queda invalidada.]  

  \InterfazFuncion{Claves}{\In{d}{diccRapido(nat, significado)}}{conj(clave)}%
  [true]
  {$res \igobs$ claves($d$)}%
  [$\Theta(1)$]
  [Devuelve el conjunto de claves del diccionario $d$]
  [$res$ es una referencia constante a un conj(clave).]

  % \InterfazFuncion{Borrar}{\In{c}{clave}, \Inout{d}{diccRapido(nat, significado)}}{}%
  % [def?($c$, $d$) $\land$ d $\igobs$ $d_{0}$]
  % {$d \igobs$ borrar($c$, $d_{0}$)}%
  % [O(N)]
  % [Borra el significado de la clave $c$]

\end{Interfaz}

\begin{Representacion}
  
  \Titulo{Representacion del DiccRapido}

  \begin{Estructura}{diccRapido(nat, significado)}[estr]
    \begin{Tupla}[estr]
      \tupItem{defs}{arreglo(conj(tuplaSignificado))}
      \tupItem{claves}{conj(nat)}
    \end{Tupla}

    \begin{Tupla}[tuplaSignificado]
      \tupItem{key}{nat}
      \tupItem{def}{significado}
    \end{Tupla}
  \end{Estructura}


  

  Invariante de representacion en castellano: \\
  \begin{enumerate}
    \item Todas las claves estan definidas en el arreglo y todas las cosas definidas estan en claves.
    \item Para toda clave $c$, la funcion de Hash convierte a $c$ en una posicion del arreglo en donde existe un elemento cuya clave es $c$.
  \end{enumerate}

  \Rep[estr][d]{
    \begin{enumerate}
      \item ( \ ($\forall$ $c$ : nat) 
      ( Pertenece?($c$, d.claves) $\iff$ 
      ( $\exists!$ i : nat ) Definido?(i, d.defs) $\yluego$ 
      ( $\exists$ s : significado ) Pertenece?($\langle$ c, s $\rangle$, d.defs[i]) ) \ ) $\yluego$

      \item ( \ ($\forall$ $c$ : nat) Pertenece?($c$, d.claves) $\implies$ (Definido?(Hash($c$, $d$), d.defs) $\yluego$ (( $\exists$ s : significado ) Pertenece?($\langle$ c, s $\rangle$, d.defs[Hash($c$, $d$)])) ) \ )
    \end{enumerate}
  }\mbox{}

  %\tadOperacion{Hash}{clave \ k, clave \ c, diccRapido(clave{,} \ significado) \ d}{nat}{}
  \tadAxioma{Hash($c, d$)}{($c$ $\%$ $\#$(d.claves))}
  % \tadAxioma{Hash($k, c, d$)}{\IF {(Definido?(fHash) $\yluego$ d.defs[fHash].key == $c$) $\lor$ ($\neg$Definido?(fHash)) } THEN fHash ELSE Hash($k+1$, $c$, $d$) FI}
  % {\small //Donde fHash es ($k$ $\%$ Cantidad(d.claves))  \\}
  % {\small //La clave $k$ es el valor sobre el cual aplico el Hash, y la clave $c$ es la clave que busco ubicar. \\}
  ~
  
  \AbsFc[estr]{Diccionario(clave, significado)}[e]{$d$ : Diccionario(clave, significado) / \\
    ($\forall$ $c$ : clave) Pertenece?($c$, e.claves) $\iff$ def?($c$, $d$) $\yluego$ \\
    ($\forall$ $c$ : clave) def?($c$, $d$) $\impluego$ \\ ( ($\forall$ $s$ : significado) s $\igobs$ obtener($c, d$) $\iff$ Pertenece?($\langle$ $c, s$ $\rangle$, d.defs[Hash($c, d$)]) )
    }

\end{Representacion}

  \begin{Algoritmos}

\textbf{Algoritmos de Agentes}
  \listofalgorithms
  \bigskip
  
  \begin{algorithm}[H]
    \text{$i$Vacio(\In{maxClaves}{nat}) $\rightarrow$ res: estr}
  
    \Begin{
      var \\
      i : nat \\

      res.claves $\leftarrow$ Vacio() \hfill \textbf{//O(1)} \\
      res.defs $\leftarrow$ CrearArreglo($maxClaves$) \hfill \textbf{//O(N)} \\
      i $\leftarrow$ 0 \hfill \textbf{//O(1)} \\
      \While(\hfill \textbf{//O(1)}){i $<$ maxClaves}
        {
          res.defs[i] $\leftarrow$ Vacio() \hfill \textbf{//O(1)} \\
          i $\leftarrow$ i + 1 \hfill \textbf{//O(1)} \\
        }
        \hfill \textbf{//While: O(N)}
    }
    \KwData{O(N) con N = maxClaves}
  
    \caption{Vacio}
  \end{algorithm}
  \rule{17.5cm}{0.4pt}
  \bigskip

  \begin{algorithm}[H]
    \text{$i$Definir(\In{k}{clave}, \In{s}{significado}, \Inout{d}{estr})}
  
    \Begin{
      var \\
      i : nat \\
      posArreglo : nat \\
      itConjunto : itConj(tuplaSignificado) \\
      encontrado : bool \\

      posArreglo $\leftarrow$ Hash($k$, $d$) \hfill \textbf{//O(1)} \\

      itConjunto $\leftarrow$ CrearIt(d.defs[posArreglo]) \hfill \textbf{//O(1)} \\
      encontrado $\leftarrow$ false \hfill \textbf{//O(1)} \\

      \While(\hfill \textbf{//O(1)}){HaySiguiente(itConjunto)}
        {
          \eIf(\hfill \textbf{//O(1)}){Siguiente(itConjunto).key $==$ $k$} 
            {
              encontrado $\leftarrow$ true \hfill \textbf{//O(1)} \\
              Siguiente(itConjunto).def $\leftarrow$ s \hfill \textbf{//O(1)} \\
            }
            {
              Avanzar(itConjunto) \hfill \textbf{//O(1)} \\
            }
        }
        \hfill \textbf{//While: O(N)}

        \If(\hfill \textbf{//O(1)}){encontrado == false}
          {
            Agregar(d.claves, $k$) \hfill \textbf{//O(1)} \\
            Agregar(d.defs[posArreglo], $\langle$ k, s $\rangle$) \hfill \textbf{//O(1)} \\
          }
      
    }
    \KwData{O(N) | O(1)}
    \KwComm{El peor caso ocurre cuando la funcion de Hash envia a todas las claves a la misma posicion del arreglo. Si se asegura buena distribucion de las claves, se obtiene un buen factor de carga, y la complejidad del $while$ se vuelve O(1) en el caso promedio (Pocos elementos en cada lugar del arreglo).}
  
    \caption{Definir}
  \end{algorithm}
  \rule{17.5cm}{0.4pt}
  \bigskip

  \begin{algorithm}[H]
    \text{$i$Def?(\In{k}{clave}, \In{d}{estr}) $\rightarrow$ res: bool}
  
    \Begin{
      res $\leftarrow$ Pertenece?(k, d.claves) \hfill \textbf{//O(N)}
    }
    \KwData{O(N)}

    \caption{Def?}
  \end{algorithm}
  \rule{17.5cm}{0.4pt}
  \bigskip

  \begin{algorithm}[H]
    \text{$i$Obtener(\In{k}{clave}, \In{d}{estr}) $\rightarrow$ res: significado}
  
    \Begin{
      var \\
      i : nat \\
      posArreglo : nat \\
      itConjunto : itConj(tuplaSignificado) \\

      posArreglo $\leftarrow$ Hash($k$, $d$) \hfill \textbf{//O(1)} \\

      itConjunto $\leftarrow$ CrearIt(d.defs[posArreglo]) \hfill \textbf{//O(1)} \\

      \While(\hfill \textbf{//O(1)}){HaySiguiente(itConjunto)}
        {
          \eIf(\hfill \textbf{//O(1)}){Siguiente(itConjunto).key $==$ $k$} 
            {
              res $\leftarrow$ Siguiente(itConjunto).def \hfill \textbf{//O(1)} \\
            }
            {
              Avanzar(itConjunto) \hfill \textbf{//O(1)} \\
            }
        }
        \hfill \textbf{//While: O(N)}
    }
    \KwData{O(N) | O(1)}
    \KwComm{El peor caso ocurre cuando la funcion de Hash envia a todas las claves a la misma posicion del arreglo. Si se asegura buena distribucion de las claves, se obtiene un buen factor de carga, y la complejidad del $while$ se vuelve O(1) en el caso promedio (Pocos elementos en cada lugar del arreglo).}
  
    \caption{Obtener}
  \end{algorithm}
  \rule{17.5cm}{0.4pt}
  \bigskip

  \begin{algorithm}[H]
    \text{$i$Claves(\In{d}{estr}) $\rightarrow$ res: conj(clave)}
  
    \Begin{
      res $\leftarrow$ d.claves \hfill \textbf{//O(1)}
    }
    \KwData{O(1)}
  
    \caption{Claves}
  \end{algorithm}
  \rule{17.5cm}{0.4pt}
  \bigskip

  \begin{algorithm}[H]
    \text{Hash(\In{c}{clave}, \In{d}{estr}) $\rightarrow$ res: nat}
  
    \Begin{
      res $\leftarrow$ (c $\%$ Cardinal(d.claves) ) \hfill \textbf{//O(1)} \\
    }
    \KwData{O(1)}
  
    \caption{Hash}
  \end{algorithm}
  \rule{17.5cm}{0.4pt}
  \bigskip

  \begin{algorithm}[H]
    \text{$i$Borrar(\In{k}{clave}, \Inout{d}{estr})}
  
    \Begin{
      var \\
      i : nat \\
      posArreglo : nat \\
      itConjunto : itConj(tuplaSignificado) \\

      posArreglo $\leftarrow$ Hash($k$, $d$) \hfill \textbf{//O(1)} \\

      itConjunto $\leftarrow$ CrearIt(d.defs[posArreglo]) \hfill \textbf{//O(1)} \\

      \While(\hfill \textbf{//O(1)}){HaySiguiente(itConjunto)}
        {
          \eIf(\hfill \textbf{//O(1)}){Siguiente(itConjunto).key $==$ $k$} 
            {
              EliminarSiguiente(itConjunto) \hfill \textbf{//O(1)} \\
            }
            {
              Avanzar(itConjunto) \hfill \textbf{//O(1)} \\
            }
        }
        \hfill \textbf{//While: O(N)}

        Eliminar(d.claves, k) \hfill \textbf{//O(N)} \\
    }
    \KwData{O(N)}
  
    \caption{Borrar}
  \end{algorithm}
  \rule{17.5cm}{0.4pt}
  \bigskip

  % \begin{algorithm}[H]
  %   \text{$i$IrA(c, d) $\rightarrow$ res: puntero(Nodo)}
  
  %   \Begin{
  %     var \\
  %     n : nat \\
  %     n $\leftarrow$ 0 \hfill \textbf{//O(1)} \\
  %     res $\leftarrow$ d.dicc \hfill \textbf{//O(1)} \\
  %     \While(\hfill \textbf{//O(1)}){n $<$ Longitud($c$)}
  %     { 
  %       res $\leftarrow$ res.sig[Ord(n)]
  %       n $\leftarrow$ n + 1 \hfill \textbf{//O(1)} \\
  %     }
  %   }
  %   \KwData{O(1)}
  
  %   \caption{nombre}
  % \end{algorithm}
  % \rule{17.5cm}{0.4pt}
  % \bigskip

  \end{Algoritmos}


\end{document}
