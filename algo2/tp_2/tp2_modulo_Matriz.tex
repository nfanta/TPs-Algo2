\documentclass[a4paper,10pt]{article}
\usepackage[paper=a4paper, hmargin=1.5cm, bottom=1.5cm, top=3.5cm]{geometry}
\usepackage[latin1]{inputenc}
\usepackage[T1]{fontenc}
\usepackage[spanish]{babel}
\usepackage{xspace}
\usepackage{xargs}
\usepackage{ifthen}
\usepackage{aed2-tad,aed2-symb,aed2-itef}
\usepackage[]{algorithm2e}

\SetKwInput{KwData}{Complejidad}
\SetKwInput{KwComm}{Comentarios}
\SetKw{ret}{return}
\SetKw{var}{var \normalfont}
\SetAlgorithmName{Algoritmo}{Algoritmo}{\large Lista de algoritmos}
\SetAlCapSkip{1em}

\newcommand{\moduloNombre}[1]{\textbf{#1}}

\let\NombreFuncion=\textsc
\let\TipoVariable=\texttt
\let\ModificadorArgumento=\textbf
\newcommand{\res}{$res$\xspace}
\newcommand{\tab}{\hspace*{7mm}}

\newcommandx{\TipoFuncion}[3]{%
  \NombreFuncion{#1}(#2) \ifx#3\empty\else $\to$ \res\,: \TipoVariable{#3}\fi%
}
\newcommand{\In}[2]{\ModificadorArgumento{in} \ensuremath{#1}\,: \TipoVariable{#2}\xspace}
\newcommand{\Out}[2]{\ModificadorArgumento{out} \ensuremath{#1}\,: \TipoVariable{#2}\xspace}
\newcommand{\Inout}[2]{\ModificadorArgumento{in/out} \ensuremath{#1}\,: \TipoVariable{#2}\xspace}
\newcommand{\Aplicar}[2]{\NombreFuncion{#1}(#2)}

\newlength{\IntFuncionLengthA}
\newlength{\IntFuncionLengthB}
\newlength{\IntFuncionLengthC}
%InterfazFuncion(nombre, argumentos, valor retorno, precondicion, postcondicion, complejidad, descripcion, aliasing)
\newcommandx{\InterfazFuncion}[9][4=true,6,7,8,9]{%
  \hangindent=\parindent
  \TipoFuncion{#1}{#2}{#3}\\%
  \textbf{Pre} $\equiv$ \{#4\}\\%
  \textbf{Post} $\equiv$ \{#5\}%
  \ifx#6\empty\else\\\textbf{Complejidad:} #6\fi%
  \ifx#7\empty\else\\\textbf{Descripci�n:} #7\fi%
  \ifx#8\empty\else\\\textbf{Aliasing:} #8\fi%
  \ifx#9\empty\else\\\textbf{Requiere:} #9\fi%
}

\newenvironment{Interfaz}{%
  \parskip=2ex%
  \noindent\textbf{\Large Interfaz}%
  \par%
}{}

\newenvironment{Representacion}{%
  \vspace*{2ex}%
  \noindent\textbf{\Large Representaci�n}%
  \vspace*{2ex}%
}{}

\newenvironment{Algoritmos}{%
  \vspace*{2ex}%
  \noindent\textbf{\Large Algoritmos}%
  \vspace*{2ex}%
}{}


\newcommand{\Titulo}[1]{
  \vspace*{1ex}\par\noindent\textbf{\large #1}\par
}

\newenvironmentx{Estructura}[2][2={estr}]{%
  \par\vspace*{2ex}%
  \TipoVariable{#1} \textbf{se representa con} \TipoVariable{#2}%
  \par\vspace*{1ex}%
}{%
  \par\vspace*{2ex}%
}%

\newboolean{EstructuraHayItems}
\newlength{\lenTupla}
\newenvironmentx{Tupla}[1][1={estr}]{%
    \settowidth{\lenTupla}{\hspace*{3mm}donde \TipoVariable{#1} es \TipoVariable{tupla}$($}%
    \addtolength{\lenTupla}{\parindent}%
    \hspace*{3mm}donde \TipoVariable{#1} es \TipoVariable{tupla}$($%
    \begin{minipage}[t]{\linewidth-\lenTupla}%
    \setboolean{EstructuraHayItems}{false}%
}{%
    $)$%
    \end{minipage}
}

\newcommandx{\tupItem}[3][1={\ }]{%
    %\hspace*{3mm}%
    \ifthenelse{\boolean{EstructuraHayItems}}{%
        ,#1%
    }{}%
    \emph{#2}: \TipoVariable{#3}%
    \setboolean{EstructuraHayItems}{true}%
}

\newcommandx{\RepFc}[3][1={estr},2={e}]{%
  \tadOperacion{Rep}{#1}{bool}{}%
  \tadAxioma{Rep($#2$)}{#3}%
}%

\newcommandx{\Rep}[3][1={estr},2={e}]{%
  \tadOperacion{Rep}{#1}{bool}{}%
  \tadAxioma{Rep($#2$)}{true \ssi #3}%
}%

\newcommandx{\Abs}[5][1={estr},3={e}]{%
  \tadOperacion{Abs}{#1/#3}{#2}{Rep($#3$)}%
  \settominwidth{\hangindent}{Abs($#3$) \igobs #4: #2 $\mid$ }%
  \addtolength{\hangindent}{\parindent}%
  Abs($#3$) \igobs #4: #2 $\mid$ #5%
}%

\newcommandx{\AbsFc}[4][1={estr},3={e}]{%
  \tadOperacion{Abs}{#1/#3}{#2}{Rep($#3$)}%
  \tadAxioma{Abs($#3$)}{#4}%
}%


\newcommand{\DRef}{\ensuremath{\rightarrow}}



%%%%%%%%%%%%%%%%%%%%%%%%%%%%
%% COMIENZO DEL DOCUMENTO %%
%%%%%%%%%%%%%%%%%%%%%%%%%%%%



\begin{document}



%%%%%%%%%%%%%%%%%%%
%% MODULO CAMPUS %%
%%%%%%%%%%%%%%%%%%%



\section{M�dulo Matriz}


\begin{Interfaz}
  
  \textbf{par�metros formales}\hangindent=2\parindent\\
  \parbox{1.7cm}{\textbf{g�nero}} $significado$\\
  \parbox[t]{1.7cm}{\textbf{funci�n}}\parbox[t]{\textwidth-2\parindent-1.7cm}{%
    \InterfazFuncion{Copiar}{\In{a}{$significado$}}{$significado$}
    {$res \igobs a$}
    [$\Theta(copy(a))$]
    [funci�n de copia de $significado$]
  }

  \textbf{se explica con}: \tadNombre{Diccionario(pos, significado)}

  \textbf{g�neros}: \TipoVariable{matriz}
  \bigskip
  \newline
  El modulo funciona como un diccionario, pero solo se utiliza con claves del tipo $pos$.
  Extiende el TAD para contemplar que la creacion de una nueva matriz requiere dos parametros, el
  $alto$ y el $ancho$.
  

  \Titulo{Operaciones b�sicas de matriz}

  \InterfazFuncion{NuevaMatriz}{\In{al}{nat}, \In{an}{nat}}{matriz}
  [true]
  {$res \igobs$ vacio}
  [$O(n * m)$]
  [Crea una nueva matriz de alto $al$ x ancho $an$.]

  \InterfazFuncion{Colocar}{\In{p}{pos}, \In{s}{significado}, \Inout{m}{matriz}}{}
  [$m \igobs m_{0}$]
  {$m \igobs$ definir($p, s, m_{0}$)}
  [$O(1)$]
  [Coloca $(define)$ el significado $s$ en la posicion $p$ de la matriz $m$.]

  \InterfazFuncion{Ocupada?}{\In{p}{pos}, \In{m}{matriz}}{bool}
  [true]
  {$res \igobs$ def?($p, m$)}
  [$O(1)$]
  [Devuelve $true$ si la posicion $p$ esta ocupada.]
  []

  \InterfazFuncion{Obtener}{\In{p}{pos}, \In{m}{matriz}}{significado}
  [def?($p, m$)]
  {$res \igobs$ obtener($p, m$)}
  [$O(1)$]
  [Retorna el $significado$ almacenado en la posicion $p$.]
  []

  \InterfazFuncion{Eliminar}{\In{p}{pos}, \Inout{m}{matriz}}{}
  [def?($p, m$) $\land$ $m \igobs m_{0}$]
  {$m \igobs$ borrar($p, m_{0}$)}
  [$O(1)$]
  [Elimina el contenido de la posicion $p$ de la matriz $m$.]
  []

  \InterfazFuncion{PosicionesOcupadas}{\In{m}{matriz}}{conj(pos)}
  [true]
  {$res \igobs$ claves($m$)}
  [$O(1)$]
  [Devuelve el conjunto de posiciones ocupadas en la matriz $m$]
  []

\end{Interfaz}
\bigskip


\begin{Representacion}
  
  \Titulo{Representaci�n de Matriz}

  \begin{Estructura}{matriz}[estr]
    \begin{Tupla}[estr]
      \tupItem{alto}{nat}
      \tupItem{ancho}{nat}
      \tupItem{claves}{conjRapido(pos)}%
      \tupItem{tablero}{vector(vector(info))}%
    \end{Tupla}

    \begin{Tupla}[info]
      \tupItem{definido}{bool}%
      \tupItem{dato}{significado}%
    \end{Tupla}

    \begin{Tupla}[pos]
      \tupItem{fila}{nat}%
      \tupItem{columna}{nat}%
    \end{Tupla}
  \end{Estructura}

  \bigskip
  \textbf{Invariante de representacion en castellano:}
  \begin{enumerate}
  
  \item La longitud de tablero es $alto$
  \item Para toda posicion de tablero, el vector que contiene posee longitud $ancho$
  \item Para toda clave $p$ en el rango de la matriz, $p$ contenida en $claves$ implica que las componentes de $c$ (.fila, .columna) en $tablero$ dan una tupla $info$ donde .definido es $true$.
  \item Analogo al anterior, pero para toda $p$ que este en el rango de la matriz y no este contenida en $claves$, la tupla $info$ posee .definido igual a $false$

  \end{enumerate}

  \bigskip
  \Rep[estr][e]{
  
  \begin{enumerate}
  \item Longitud(tablero) $\igobs$ alto $\land$
  \item ($\forall$ $i$ : int) ($i$ < Longitud(tablero)) $\impluego$ Longitud(tablero[$i$]) $\igobs$ ancho $\yluego$
  \item ($\forall$ $p$ : pos) ( $p$.fila $\leq$ alto $\land$ $p$.columna $\leq$ ancho $\land$ $p \in$ e.claves ) $\impluego$ \\ (talbero[$p$.fila][$p$.columna].definido $\igobs$ $true$)
  \item ($\forall$ $p$ : pos) ( $p$.fila $\leq$ alto $\land$ $p$.columna $\leq$ ancho $\land$ $p \notin$ e.claves ) $\impluego$ \\ (talbero[$p$.fila][$p$.columna].definido $\igobs$ $false$)
  \end{enumerate}
  
  }\mbox{}

  \bigskip
  \AbsFc[estr]{dicc($pos, significado$)}[e]{m : dicc($pos, significado$) \\
    ($\forall$ $p$ : pos) def?($p$, e.claves) $\igobs$ def?($p, m$) $\land$ \\
    ($\forall$ $p$ : pos) def?($p$, e.claves) $\impluego$ $\Pi_{2}$(obtener($p$, e)) $\igobs$ obtener($p, m$))
  }
  \smallskip
  \textit{\scriptsize Las claves definidas y sus significados son iguales}
  \bigskip

\end{Representacion}

\begin{Algoritmos}

  % \textbf{Algoritmos de Campus}
  \listofalgorithms
  \newpage
  

  \begin{algorithm}[H]
    \text{$i$NuevaMatriz(\In{al}{nat}, \In{an}{nat}) $\rightarrow$ res: estr}
  
    \Begin{
      res.alto $\leftarrow$ al \hfill \textbf{//O(1)}\\
      res.ancho $\leftarrow$ an \hfill \textbf{//O(1)}\\
      res.claves $\leftarrow$ Vacio() \hfill \textbf{//O(1)}\\
      res.tablero $\leftarrow$ CrearArreglo(al) \hfill \textbf{//O(al)}\\
      \For(\hfill \textbf{//O(al)}){i $\leftarrow$ 0..($al - 1$)}{
        res.tablero[i] $\leftarrow$ CrearArreglo(an) \hfill \textbf{//O(an)}
      }
    }
    \KwData{$O(al * an)$}
  
    \caption{NuevaMatriz}
  \end{algorithm}
  \rule{17.5cm}{0.4pt}
  \bigskip



  \begin{algorithm}[H]
    \text{$i$Colocar(\In{p}{pos}, \In{s}{significado}, \Inout{e}{estr})}
  
    \Begin{
      \If(\hfill \textbf{//O(1)}){p.fila > e.alto $\lor$ p.columna > e.ancho}
      {
        \ret{e}\\
        \Else{
          Agregar(p, e.claves) \hfill \textbf{//O(1)} \\
          e.tablero[p.fila][p.columna] $\leftarrow$ <$true$, s> \hfill \textbf{//O(1)}
        }
      }
    }
    \KwData{O(1)}
  
    \caption{Colocar}
  \end{algorithm}
  \rule{17.5cm}{0.4pt}
  \bigskip


  \begin{algorithm}[H]
    \text{$i$Ocupada?(\In{p}{pos}, \In{e}{estr}) $\rightarrow$ res: bool}
  
    \Begin{
      \If(\hfill \textbf{//O(1)}){p.fila > e.alto $\lor$ p.columna > e.ancho}
      {
        res $\leftarrow$ false \hfill \textbf{//O(1)} \\
        \Else{
          res $\leftarrow$ $\Pi_{1}$(e.tablero[p.fila][p.columna]) \hfill \textbf{//O(1)} \\
        }
      }
      \ret{res}
    }
    \KwData{O(1)}
  
    \caption{Ocupada?}
  \end{algorithm}
  \rule{17.5cm}{0.4pt}
  \bigskip


  \begin{algorithm}[H]
    \text{$i$Obtener(\In{p}{pos}, \In{e}{estr}) $\rightarrow$ res: significado}
  
    \Begin{
      res $\leftarrow$ $\Pi_{2}$(e.tablero[p.fila][p.columna]) \hfill \textbf{//O(1)} \\
      \ret{res}
    }
    \KwData{O(1)}
  
    \caption{Obtener}
  \end{algorithm}
  \rule{17.5cm}{0.4pt}
  \bigskip


  \begin{algorithm}[H]
    \text{$i$Eliminar(\In{p}{pos}, \Inout{e}{estr})}
  
    \Begin{
      Eliminar(p, e.claves) \hfill \textbf{//O(lo que diga mati)}\\
      $\Pi_{1}$(e.tablero[p.alto][p.columna]) $\leftarrow$ $false$ \hfill \textbf{//O(1)} \\
    }
    \KwData{O(1)}
  
    \caption{Eliminar}
  \end{algorithm}
  \rule{17.5cm}{0.4pt}
  \bigskip


  \begin{algorithm}[H]
    \text{$i$PosicionesOcupadas(\In{e}{estr}) $\rightarrow$ res: conj(pos)}
  
    \Begin{
      res $\leftarrow$ e.claves \hfill \textbf{//O(1)}\\
      \ret{res}
    }
    \KwData{O(1)}
  
    \caption{PosicionesOcupadas}
  \end{algorithm}
  \rule{17.5cm}{0.4pt}
  \bigskip




\end{Algoritmos}

\end{document}
