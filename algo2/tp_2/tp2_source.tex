\documentclass[a4paper,10pt]{article}
\usepackage[paper=a4paper, hmargin=1.5cm, bottom=1.5cm, top=3.5cm]{geometry}
\usepackage[latin1]{inputenc}
\usepackage[T1]{fontenc}
\usepackage[spanish]{babel}
\usepackage{xspace}
\usepackage{xargs}
\usepackage{ifthen}
\usepackage{aed2-tad,aed2-symb,aed2-itef}
\usepackage[]{algorithm2e}

\SetKwInput{KwData}{Complejidad}
\SetAlgorithmName{Algoritmo}{Algoritmo}{Lista de algoritmos}
\SetAlCapSkip{1em}

\newcommand{\moduloNombre}[1]{\textbf{#1}}

\let\NombreFuncion=\textsc
\let\TipoVariable=\texttt
\let\ModificadorArgumento=\textbf
\newcommand{\res}{$res$\xspace}
\newcommand{\tab}{\hspace*{7mm}}

\newcommandx{\TipoFuncion}[3]{%
  \NombreFuncion{#1}(#2) \ifx#3\empty\else $\to$ \res\,: \TipoVariable{#3}\fi%
}
\newcommand{\In}[2]{\ModificadorArgumento{in} \ensuremath{#1}\,: \TipoVariable{#2}\xspace}
\newcommand{\Out}[2]{\ModificadorArgumento{out} \ensuremath{#1}\,: \TipoVariable{#2}\xspace}
\newcommand{\Inout}[2]{\ModificadorArgumento{in/out} \ensuremath{#1}\,: \TipoVariable{#2}\xspace}
\newcommand{\Aplicar}[2]{\NombreFuncion{#1}(#2)}

\newlength{\IntFuncionLengthA}
\newlength{\IntFuncionLengthB}
\newlength{\IntFuncionLengthC}
%InterfazFuncion(nombre, argumentos, valor retorno, precondicion, postcondicion, complejidad, descripcion, aliasing)
\newcommandx{\InterfazFuncion}[9][4=true,6,7,8,9]{%
  \hangindent=\parindent
  \TipoFuncion{#1}{#2}{#3}\\%
  \textbf{Pre} $\equiv$ \{#4\}\\%
  \textbf{Post} $\equiv$ \{#5\}%
  \ifx#6\empty\else\\\textbf{Complejidad:} #6\fi%
  \ifx#7\empty\else\\\textbf{Descripci�n:} #7\fi%
  \ifx#8\empty\else\\\textbf{Aliasing:} #8\fi%
  \ifx#9\empty\else\\\textbf{Requiere:} #9\fi%
}

\newenvironment{Interfaz}{%
  \parskip=2ex%
  \noindent\textbf{\Large Interfaz}%
  \par%
}{}

\newenvironment{Representacion}{%
  \vspace*{2ex}%
  \noindent\textbf{\Large Representaci�n}%
  \vspace*{2ex}%
}{}

\newenvironment{Algoritmos}{%
  \vspace*{2ex}%
  \noindent\textbf{\Large Algoritmos}%
  \vspace*{2ex}%
}{}


\newcommand{\Titulo}[1]{
  \vspace*{1ex}\par\noindent\textbf{\large #1}\par
}

\newenvironmentx{Estructura}[2][2={estr}]{%
  \par\vspace*{2ex}%
  \TipoVariable{#1} \textbf{se representa con} \TipoVariable{#2}%
  \par\vspace*{1ex}%
}{%
  \par\vspace*{2ex}%
}%

\newboolean{EstructuraHayItems}
\newlength{\lenTupla}
\newenvironmentx{Tupla}[1][1={estr}]{%
    \settowidth{\lenTupla}{\hspace*{3mm}donde \TipoVariable{#1} es \TipoVariable{tupla}$($}%
    \addtolength{\lenTupla}{\parindent}%
    \hspace*{3mm}donde \TipoVariable{#1} es \TipoVariable{tupla}$($%
    \begin{minipage}[t]{\linewidth-\lenTupla}%
    \setboolean{EstructuraHayItems}{false}%
}{%
    $)$%
    \end{minipage}
}

\newcommandx{\tupItem}[3][1={\ }]{%
    %\hspace*{3mm}%
    \ifthenelse{\boolean{EstructuraHayItems}}{%
        ,#1%
    }{}%
    \emph{#2}: \TipoVariable{#3}%
    \setboolean{EstructuraHayItems}{true}%
}

\newcommandx{\RepFc}[3][1={estr},2={e}]{%
  \tadOperacion{Rep}{#1}{bool}{}%
  \tadAxioma{Rep($#2$)}{#3}%
}%

\newcommandx{\Rep}[3][1={estr},2={e}]{%
  \tadOperacion{Rep}{#1}{bool}{}%
  \tadAxioma{Rep($#2$)}{true \ssi #3}%
}%

\newcommandx{\Abs}[5][1={estr},3={e}]{%
  \tadOperacion{Abs}{#1/#3}{#2}{Rep($#3$)}%
  \settominwidth{\hangindent}{Abs($#3$) \igobs #4: #2 $\mid$ }%
  \addtolength{\hangindent}{\parindent}%
  Abs($#3$) \igobs #4: #2 $\mid$ #5%
}%

\newcommandx{\AbsFc}[4][1={estr},3={e}]{%
  \tadOperacion{Abs}{#1/#3}{#2}{Rep($#3$)}%
  \tadAxioma{Abs($#3$)}{#4}%
}%


\newcommand{\DRef}{\ensuremath{\rightarrow}}

\begin{document}

\section{M�dulo Campus}


\begin{Interfaz}
  
  \textbf{se explica con}: \tadNombre{Campus}

  \textbf{g�neros}: \TipoVariable{campus}

  \Titulo{Operaciones b�sicas de campus}

  \InterfazFuncion{NuevoCampus}{\In{al}{nat}, \In{an}{nat}}{campus}%
  [true]
  {$res \igobs$ crearCampus($al, an$)}%
  [$\Theta(1)$]
  [Crea un nuevo campus vacio de alto $al$ x ancho $an$.]

  \InterfazFuncion{AgregarObstaculo}{\In{p}{pos)}, \Inout{c}{campus}}{}
  [posValida($p, c$) $\land$ $\neg$ocupada?($p,c$) $\land$ $c \igobs c_{0}$ ]
  {$c$ $\igobs$ agregarObstaculo($p, c_{0}$)}
  []
  [Agrega un obstaculo al campus $c$ en la posicion $p$.]
  []

  \InterfazFuncion{Filas}{\In{c}{campus}}{nat}
  [true]
  {$res \igobs$ filas($c$}
  []
  [Devuelve el alto (filas) del campus $c$.]
  []

  \InterfazFuncion{Columnas}{\In{c}{campus}}{nat}
  [true]
  {$res \igobs$ columnas($c$}
  []
  [Devuelve el ancho (columnas) del campus $c$.]
  []

  \InterfazFuncion{Ocupada?}{\In{p}{pos}, \In{c}{campus}}{bool}
  [true]
  {$res \igobs$ posValida?($p, c$)}
  []
  [Devuelve $true$ si la posicion $p$ es valida en el campus $c$, sino retorna $false$.]
  []

  \InterfazFuncion{EsIngreso?}{\In{p}{pos}, \In{c}{campus}}{bool}
  [true]
  {$res \igobs$ esIngreso?($p, c$)}
  []
  [Verifica si la posicion $p$ es una entrada del campus $c$.]
  []

  \InterfazFuncion{IngresoSuperior?}{\In{p}{pos}, \In{c}{campus}}{bool}
  [true]
  {$res \igobs$ ingresoSuperior?($p, c$)}
  []
  [Verifica si la posicion $p$ es una entrada superior del campus $c$.]
  []

  \InterfazFuncion{IngresoInferior?}{\In{p}{pos}, \In{c}{campus}}{bool}
  [true]
  {$res \igobs$ ingresoInferior?($p, c$)}
  []
  [Verifica si la posicion $p$ es una entrada inferior del campus $c$.]
  []

  \InterfazFuncion{Vecinos}{\In{p}{pos}, \In{c}{campus}}{conj($pos$)}
  [posValida($p, c$)]
  {$res \igobs$ vecinos($p, c$)}
  []
  [Devuelve un conjunto de las posiciones que rodean a $p$ en el campus $c$]
  []

  \InterfazFuncion{Distancia}{ \In{p_{0}}{pos}, \In{p_{1}}{pos}, \In{c}{campus}}{nat}
  [true]
  {$res \igobs$ distancia($p_{0}$, $p_{1}$, $c$)}
  []
  [Devuelve la distancia, en casilleros, desde la posicion $p_{0}$ a la posicion $p_{1}$.]
  []

  \InterfazFuncion{ProxPosicion}{\In{p}{pos}, \In{d}{dir}, \In{c}{campus}}{pos}
  [posValida($p, c$)]
  {$res \igobs$ proxPosicion($p, d, c$)}
  []
  [Indica la posicion que se encuentra al lado de $p$, en la direccion $d$.]
  []

  \InterfazFuncion{Obstaculos}{\In{c}{campus}}{conj($pos$)}
  [true]
  {$res \igobs$ obstaculos($c$)}
  []
  [Devuelve un conjunto que contiene todas las posiciones ocupadas por obstaculos en el campus $c$. EXTENDER TAD.]


\end{Interfaz}

\begin{Representacion}
  
  \Titulo{Representaci�n de la lista}

  \begin{Estructura}{campus}[estr]
    \begin{Tupla}[estr]
      \tupItem{alto}{nat}%
      \tupItem{ancho}{nat}%
      \tupItem{obstaculos}{matriz(bool)}%
    \end{Tupla}

    \begin{Tupla}[pos]
      \tupItem{fila}{nat}%
      \tupItem{columna}{nat}%
    \end{Tupla}
  \end{Estructura}

  Invariante de representacion en castellano:
  \begin{enumerate}
  
  \item Para toda $p$ de tipo $pos$, si $p$ esta definida en obstaculos, entonces tanto la fila como la columna de $p$ son menores o iguales a $alto$ y $ancho$ respectivamente.
  
  \end{enumerate}

  \Rep[estr][e]{
  
  \begin{enumerate}
  \item ($\forall$ $p$ : pos) $p$ $\in$ claves(e.obstaculos) $\implies$ ( p.fila $\leq$ c.alto $\land$ p.columna $\leq$ c.ancho )
  \end{enumerate}
  
  }\mbox{}

  \AbsFc[estr]{campus}[e]{c : campus / \\
    ($\forall$ $p$ : pos) def?($p$, e.obstaculos) $\igobs$ ocupada?($p$, c) $\land$ 
    alto($c$) $\igobs$ e.alto $\land$
    ancho($c$) $\igobs$ e.ancho}

\end{Representacion}

  \begin{Algoritmos}

\textbf{Algoritmos de Campus}
  \listofalgorithms
    
  \begin{algorithm}[h]
    \text{$i$NuevoCampus(\In{al}{nat}, \In{an}{nat}) $\rightarrow$ res: estr}

    \Begin{
      res.computadoras $\leftarrow $ vacio() \hfill \textbf{//O(1)}\\
      res.mapa $\leftarrow $ vacia() \hfill \textbf{//O(1)}\\
      res.indexToString $\leftarrow $ vacia() \hfill \textbf{//O(1)}
    }
    \KwData{O(1)}

    \caption{NuevoCampus}    
  \end{algorithm}

  \begin{algorithm}[h]
    \text{$i$AgregarObstaculo(\In{p}{pos}, \Inout{e}{estr})}

    \Begin{
      Colocar($p$, $true$, e.obstaculos) \hfill \textbf{//O(1)}
    }
    \KwData{O(1)}

    \caption{AgregarObstaculo}
  \end{algorithm}

  \end{Algoritmos}

\end{document}
