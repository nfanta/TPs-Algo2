\documentclass[a4paper,10pt]{article}
\usepackage[paper=a4paper, hmargin=1.5cm, bottom=1.5cm, top=3.5cm]{geometry}
\usepackage[latin1]{inputenc}
\usepackage[T1]{fontenc}
\usepackage[spanish]{babel}
\usepackage{xspace}
\usepackage{xargs}
\usepackage{ifthen}
\usepackage{aed2-tad,aed2-symb,aed2-itef}
\usepackage[]{algorithm2e}

\SetKwInput{KwData}{Complejidad}
\SetKwInput{KwComm}{Comentarios}
\SetKwInput{KwExp}{Explicacion}
\SetAlgorithmName{Algoritmo}{Algoritmo}{Lista de algoritmos}
\SetAlCapSkip{1em}


\newcommand{\moduloNombre}[1]{\textbf{#1}}

\let\NombreFuncion=\textsc
\let\TipoVariable=\texttt
\let\ModificadorArgumento=\textbf
\newcommand{\res}{$res$\xspace}
\newcommand{\tab}{\hspace*{7mm}}

\newcommandx{\TipoFuncion}[3]{%
  \NombreFuncion{#1}(#2) \ifx#3\empty\else $\to$ \res\,: \TipoVariable{#3}\fi%
}
\newcommand{\In}[2]{\ModificadorArgumento{in} \ensuremath{#1}\,: \TipoVariable{#2}\xspace}
\newcommand{\Out}[2]{\ModificadorArgumento{out} \ensuremath{#1}\,: \TipoVariable{#2}\xspace}
\newcommand{\Inout}[2]{\ModificadorArgumento{in/out} \ensuremath{#1}\,: \TipoVariable{#2}\xspace}
\newcommand{\Aplicar}[2]{\NombreFuncion{#1}(#2)}

\newlength{\IntFuncionLengthA}
\newlength{\IntFuncionLengthB}
\newlength{\IntFuncionLengthC}
%InterfazFuncion(nombre, argumentos, valor retorno, precondicion, postcondicion, complejidad, descripcion, aliasing)
\newcommandx{\InterfazFuncion}[9][4=true,6,7,8,9]{%
  \hangindent=\parindent
  \TipoFuncion{#1}{#2}{#3}\\%
  \textbf{Pre} $\equiv$ \{#4\}\\%
  \textbf{Post} $\equiv$ \{#5\}%
  \ifx#6\empty\else\\\textbf{Complejidad:} #6\fi%
  \ifx#7\empty\else\\\textbf{Descripci�n:} #7\fi%
  \ifx#8\empty\else\\\textbf{Aliasing:} #8\fi%
  \ifx#9\empty\else\\\textbf{Requiere:} #9\fi%
}

\newenvironment{Interfaz}{%
  \parskip=2ex%
  \noindent\textbf{\Large Interfaz}%
  \par%
}{}

\newenvironment{Representacion}{%
  \vspace*{2ex}%
  \noindent\textbf{\Large Representaci�n}%
  \vspace*{2ex}%
}{}

\newenvironment{Algoritmos}{%
  \vspace*{2ex}%
  \noindent\textbf{\Large Algoritmos}%
  \vspace*{2ex}%
}{}


\newcommand{\Titulo}[1]{
  \vspace*{1ex}\par\noindent\textbf{\large #1}\par
}

\newenvironmentx{Estructura}[2][2={estr}]{%
  \par\vspace*{2ex}%
  \TipoVariable{#1} \textbf{se representa con} \TipoVariable{#2}%
  \par\vspace*{1ex}%
}{%
  \par\vspace*{2ex}%
}%

\newboolean{EstructuraHayItems}
\newlength{\lenTupla}
\newenvironmentx{Tupla}[1][1={estr}]{%
    \settowidth{\lenTupla}{\hspace*{3mm}donde \TipoVariable{#1} es \TipoVariable{tupla}$($}%
    \addtolength{\lenTupla}{\parindent}%
    \hspace*{3mm}donde \TipoVariable{#1} es \TipoVariable{tupla}$($%
    \begin{minipage}[t]{\linewidth-\lenTupla}%
    \setboolean{EstructuraHayItems}{false}%
}{%
    $)$%
    \end{minipage}
}

\newcommandx{\tupItem}[3][1={\ }]{%
    %\hspace*{3mm}%
    \ifthenelse{\boolean{EstructuraHayItems}}{%
        ,#1%
    }{}%
    \emph{#2}: \TipoVariable{#3}%
    \setboolean{EstructuraHayItems}{true}%
}

\newcommandx{\RepFc}[3][1={estr},2={e}]{%
  \tadOperacion{Rep}{#1}{bool}{}%
  \tadAxioma{Rep($#2$)}{#3}%
}%

\newcommandx{\Rep}[3][1={estr},2={e}]{%
  \tadOperacion{Rep}{#1}{bool}{}%
  \tadAxioma{Rep($#2$)}{true \ssi #3}%
}%

\newcommandx{\Abs}[5][1={estr},3={e}]{%
  \tadOperacion{Abs}{#1/#3}{#2}{Rep($#3$)}%
  \settominwidth{\hangindent}{Abs($#3$) \igobs #4: #2 $\mid$ }%
  \addtolength{\hangindent}{\parindent}%
  Abs($#3$) \igobs #4: #2 $\mid$ #5%
}%

\newcommandx{\AbsFc}[4][1={estr},3={e}]{%
  \tadOperacion{Abs}{#1/#3}{#2}{Rep($#3$)}%
  \tadAxioma{Abs($#3$)}{#4}%
}%


\newcommand{\DRef}{\ensuremath{\rightarrow}}

\begin{document}


\section{M�dulo Conjunto($\alpha$)}


\begin{Interfaz}
  
  \textbf{se explica con}: \tadNombre{Conjunto($\alpha$)}

  \textbf{g�neros}: \TipoVariable{conjRapido($\alpha$)}

  \Titulo{Operaciones b�sicas de conjunto}

  \InterfazFuncion{Vacio}{}{conjRapido($\alpha$)}%
  [true]
  {$res \igobs$ $\emptyset$}%
  [$\Theta(1)$]
  [Crea un nuevo conjunto vacio]

  \InterfazFuncion{Agregar}{\In{a}{$\alpha$}\Inout{c}{conjRapido($\alpha$)}}{}%
  [c $\igobs$ $c_{0}$]
  {$c \igobs$ Ag($a$, $c_{0}$)}%
  [$\Theta(1)$]
  [Agrega el elemento $a$ al conjunto $c$]

  \InterfazFuncion{Vacio?}{\In{c}{conjRapido($\alpha$)}}{bool}%
  [true]
  {$res \igobs$ $\emptyset$?(c)}%
  [$\Theta(1)$]
  [Devuelve $true$ si el conjunto esta vacio]

  \InterfazFuncion{En}{\In{a}{$\alpha$}, \In{c}{conjRapido($\alpha$)}}{bool}%
  [true]
  {$res \igobs$ $a$ $\in$ c}%
  [O(N)]
  [Devuelve $true$ si el elemento $a$ pertenece al conjunto $c$]

  \InterfazFuncion{Cantidad}{\In{c}{conjRapido($\alpha$)}}{nat}%
  [true]
  {$res \igobs$ $\#$c}%
  [O($N^{2}$)]
  [Devuelve la cantidad de elementos definidos en $c$. En este conjunto se sacrifica la complejidad de $Cantidad$ por $Agregar$ siempre en O(1)]  

  \InterfazFuncion{DameUno}{\In{c}{conjRapido($\alpha$)}}{$\alpha$}%
  [$\neg$$\emptyset$?($c$)]
  {$res \igobs$ dameUno($c$)}%
  [$\Theta(1)$]
  [Devuelve un elemento del conjunto.]

  \InterfazFuncion{SinUno}{\Inout{c}{conjRapido($\alpha$)}}{}%
  [$\neg$$\emptyset$?($c$) $\land$ $c \igobs c_{0}$]
  {$c \igobs$ sinUno($c_{0}$)}%
  [O(N)]
  [Devuelve el conjunto sin el elemento que devuelve DameUno($c$).]

  \InterfazFuncion{Borrar}{\In{a}{$\alpha$}, \Inout{c}{conjRapido($\alpha$)}}{}%
  [$a$ $\in$ $c$ $\land$ $c \igobs c_{0}$]
  {$c \igobs$ $c_{0}$ - \{ $a$ \}}%
  [O(N)]
  [Devuelve el conjunto sin el elemento $a$.]

\end{Interfaz}

\begin{Representacion}
  
  \Titulo{Representacion del Conjunto}

  \begin{Estructura}{conjRapido($\alpha$)}[lista($\alpha$)]
  \end{Estructura}

  Invariante de representacion en castellano: \\
  {\small Cualquier lista de elementos de tipo $\alpha$ es un conjunto valido de tipo $\alpha$ \\}

  \Rep[lista($\alpha$)][l]{true}\mbox{}
  
  \AbsFc[lista($\alpha$)]{conjRapido($\alpha$)}[c]{$con$ : conj($\alpha$) / \\
    ($\forall$ $a$ : $\alpha$) $a$ $\in$ $con$ $\iff$ En($a$, $c$)
    }

\end{Representacion}

  \begin{Algoritmos}

\textbf{Algoritmos de Agentes}
  \listofalgorithms
  
  \begin{algorithm}[H]
    \text{$i$Vacio() $\rightarrow$ res: lista($\alpha$)}
  
    \Begin{
      res $\leftarrow$ Vacia() \hfill \textbf{//O(1)}
    }
    \KwData{O(1)}
  
    \caption{Vacio}
  \end{algorithm}
  \rule{17.5cm}{0.4pt}
  \bigskip

  \begin{algorithm}[H]
    \text{$i$Agregar(\In{a}{$\alpha$}, \Inout{c}{lista($\alpha$)})}
  
    \Begin{
      AgregarAtras($c, a$) \hfill \textbf{//O(1)}
    }
    \KwData{O(1)}
  
    \caption{Agregar}
  \end{algorithm}
  \rule{17.5cm}{0.4pt}
  \bigskip

  \begin{algorithm}[H]
    \text{$i$Vacio?(\In{c}{lista($\alpha$)}) $\rightarrow$ res: bool}
  
    \Begin{
      Vacia?(c) \hfill \textbf{//O(1)}
    }
    \KwData{O(1)}
  
    \caption{Vacio?}
  \end{algorithm}
  \rule{17.5cm}{0.4pt}
  \bigskip

  \begin{algorithm}[H]
    \text{$i$En(\In{a}{$\alpha$}, \In{c}{lista($\alpha$)}) $\rightarrow$ res: bool}
  
    \Begin{
      var \\
      iterador : itLista \\
      iterador $\leftarrow$ CrearIt($l$) \hfill \textbf{//O(1)} \\
      res $\leftarrow$ false \hfill \textbf{//O(1)} \\
      \While(\hfill \textbf{//O(1)}) {HaySiguiente(iterador)}
      {
        \If(\hfill \textbf{//O(1)}){Siguiente(iterador) == $a$}
          {
            res $\leftarrow$ true \hfill \textbf{//O(1)}
          }
        Avanzar(iterador) \hfill \textbf{//O(1)}
      }
      \hfill \textbf{//While: O(N)}
    }
    \KwData{O(N)}
  
    \caption{En}
  \end{algorithm}
  \rule{17.5cm}{0.4pt}
  \bigskip

  \begin{algorithm}[H]
    \text{$i$DameUno(\In{c}{lista($\alpha$)}) $\rightarrow$ res: $\alpha$}
  
    \Begin{
      Primero(c) \hfill \textbf{//O(1)}
    }
    \KwData{O(1)}
  
    \caption{DameUno}
  \end{algorithm}
  \rule{17.5cm}{0.4pt}
  \bigskip

  \begin{algorithm}[H]
    \text{$i$SinUno(\Inout{c}{lista($\alpha$)})}
  
    \Begin{
      Borrar(DameUno($c$), $c$) \hfill \textbf{//O(N)}
    }
    \KwData{O(N)}
  
    \caption{SinUno}
  \end{algorithm}
  \rule{17.5cm}{0.4pt}
  \bigskip

  \begin{algorithm}[H]
    \text{$i$Cantidad(\In{c}{lista($\alpha$)}) $\rightarrow$ res: nat}
  
    \Begin{
      var \\
      iterador : itLista($\alpha$) \\
      copiaLista : lista($\alpha$) \\
      cantidad : nat \\
      cantidad $\leftarrow$ 0 \hfill \textbf{//O(1)} \\
      iterador $\leftarrow$ CrearIt(c) \hfill \textbf{//O(1)} \\
      res $\leftarrow$ false \hfill \textbf{//O(1)} \\
      copiaLista $\leftarrow$ Copiar(c) \hfill \textbf{//O(N)} \\

      \While(\hfill \textbf{//O(1)}) {HaySiguiente(iterador)}
      {
        Fin(copiaLista) \hfill \textbf{//O(1)} \\
        \If(\hfill \textbf{//O(N)}){$\neg$En(Siguiente(iterador), copiaLista)}
          {
            cantidad $\leftarrow$ cantidad + 1 \hfill \textbf{//O(1)} \\
          }
        Avanzar(iterador) \hfill \textbf{//O(1)}
      }
      \hfill \textbf{//While: O($\sum_{i=1}^{n} n$)}
    }
    \KwData{O($N^{2}$)}
    \KwComm{O($\sum_{i=1}^{n} n$) = O($\frac{1}{2}n(n+1)$) = O($\frac{1}{2}n^{2}+\frac{1}{2}n$) = O($n^{2}$)}
  
    \caption{Cantidad}
  \end{algorithm}
  \rule{17.5cm}{0.4pt}
  \bigskip

  \begin{algorithm}[H]
    \text{$i$Borrar(\In{a}{$\alpha$}, \Inout{c}{lista($\alpha$)})}
  
    \Begin{
      var \\
      iterador : itLista($\alpha$) \\
      iterador $\leftarrow$ CrearIt(c) \hfill \textbf{//O(1)} \\
      \While(\hfill \textbf{//O(1)}){HaySiguiente(iterador)}
      {
        \eIf(\hfill \textbf{//O(1)}){Siguiente(iterador) == $a$}
          {
            EliminarSiguiente(iterador) \hfill \textbf{//O(1)} \\
          }
          {
            Avanzar(iterador) \hfill \textbf{//O(1)} \\
          }
      }
      \hfill \textbf{//While: O(N)}
    }
    \KwData{O(N)}
  
    \caption{Borrar}
  \end{algorithm}
  \rule{17.5cm}{0.4pt}
  \bigskip

  \end{Algoritmos}


\end{document}
